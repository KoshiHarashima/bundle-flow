% ==== article 用(upLaTeX + dvipdfmx 前提)====
% Overleaf/latexmk で uplatex→dvipdfmx を想定
% エンジンはクラスやパッケージで自動判別しない(手元設定で固定)

% --- 数式・基本 ---
\usepackage{amsmath,amssymb,amsthm,mathtools,bm}
\allowdisplaybreaks
\numberwithin{equation}{section}

% --- 図とハイパーリンク(dvipdfmx明示)---
\usepackage[dvipdfmx]{graphicx}
\usepackage[dvipdfmx,hidelinks,bookmarks=true,bookmarksnumbered=true]{hyperref}
\usepackage[nameinlink,capitalise]{cleveref}

% --- 表(経済論文向け)---
\usepackage{booktabs,threeparttable,threeparttablex}
\usepackage{tabularx,array}
\renewcommand{\arraystretch}{1.1}

% --- 色・箇条書き ---
\usepackage{xcolor}
\usepackage[shortlabels]{enumitem}
\setlist{noitemsep,topsep=2pt}

% --- キャプション ---
\usepackage{caption}
\usepackage{subcaption}

% --- 日本語(upLaTeX)---
\usepackage{otf}
\usepackage[noalphabet]{pxchfon}
\renewcommand{\kanjifamilydefault}{\gtdefault}
% HaranoAji が無い環境では次行をコメントアウト
\IfFileExists{HaranoAjiGothic-Medium.otf}{%
  \setboldgothicfont{HaranoAjiGothic-Medium.otf}%
}{}

% 図の相対パス(article配下の figures を既定に)
\graphicspath{{./figures/}}

% ==== 定理環境(節ごと番号)====
\theoremstyle{plain}
\newtheorem{theorem}{Theorem}[section]
\newtheorem{lemma}[theorem]{Lemma}
\newtheorem{proposition}[theorem]{Proposition}
\newtheorem{corollary}[theorem]{Corollary}
\theoremstyle{definition}
\newtheorem{definition}[theorem]{Definition}
\newtheorem{assumption}[theorem]{Assumption}
\newtheorem{problem}[theorem]{Problem}
\theoremstyle{remark}
\newtheorem{remark}[theorem]{Remark}

% cleveref の表示名
\crefname{theorem}{Theorem}{Theorems}
\crefname{lemma}{Lemma}{Lemmas}
\crefname{proposition}{Proposition}{Propositions}
\crefname{corollary}{Corollary}{Corollaries}
\crefname{definition}{Definition}{Definitions}
\crefname{assumption}{Assumption}{Assumptions}
\crefname{problem}{Problem}{Problems}
\crefname{remark}{Remark}{Remarks}
\crefname{equation}{Eq.}{Eqs.}
\crefname{figure}{Fig.}{Figs.}
\crefname{table}{Table}{Tables}
\crefname{section}{Section}{Sections}

%======数学記法===============
\DeclareMathOperator*{\argmax}{arg\,max}


% ---- 文献(biblatex + biber 前提;Paperpileの .bib を主ファイル基準で)----
\usepackage{csquotes}
\usepackage[
  backend=biber,
  style=apa,      % 必要に応じて phys/authoryear へ差替え
  natbib=true
]{biblatex}
\addbibresource{paperpile.bib}

