\documentclass[dvipdfmx,autodetect-engine]{beamer}

% 好みのテーマ(必要なら後で変更)
\usetheme{Madrid}
\usecolortheme{default}

% 共通前置き&マクロ
% ==== beamer 用(upLaTeX + dvipdfmx 前提)====

% 数式
\usepackage{amsmath,amssymb,mathtools,bm}
\allowdisplaybreaks

% 図(beamerはgraphicxを内部導入するが,dvipdfmx用途で明示してもよい)
\usepackage[dvipdfmx]{graphicx}

% ハイパーリンク:beamerがhyperrefを読むので,細部設定のみ
\hypersetup{
  hidelinks,
  pdfpagemode=UseNone,
  bookmarksnumbered=true
}

% ---- 文献(biblatex + biber 前提;Paperpileの .bib を主ファイル基準で)----
\usepackage{csquotes}
\usepackage[
  backend=biber,
  style=apa,      % 必要に応じて phys/authoryear へ差替え
  natbib=true
]{biblatex}
\addbibresource{paperpile.bib}

% 参照(定理名や図表名のスマート参照)
\usepackage[nameinlink,capitalise]{cleveref}

% 表(論文向けの見栄え)
\usepackage{booktabs,threeparttable,threeparttablex}
\usepackage{tabularx,array}
\renewcommand{\arraystretch}{1.1}

% 色
\usepackage{xcolor}

% 日本語(upLaTeX)
\usepackage[deluxe]{otf}
\usepackage[noalphabet]{pxchfon}
\renewcommand{\kanjifamilydefault}{\gtdefault}
\IfFileExists{HaranoAjiGothic-Medium.otf}{%
  \setboldgothicfont{HaranoAjiGothic-Medium.otf}%
}{}

% 図の相対パス
\graphicspath{{./figures/}}




% ==== メタ情報(必要に応じて毎回ここだけ更新)====
\newcommand{\papertitle}{N}
\newcommand{\paperauthor}{Koshi Harashima}
\newcommand{\paperaffil}{Northwestern University}
\newcommand{\paperdate}{October 6, 2025}

% ==== 数式系のマクロ ====
\newcommand*{\N}{\mathbb{N}}
\newcommand*{\Z}{\mathbb{Z}}
\newcommand*{\Q}{\mathbb{Q}}
\newcommand*{\R}{\mathbb{R}}
\newcommand*{\C}{\mathbb{C}}
\newcommand*{\I}{\mathbb{I}}

\newcommand*{\boldalpha}  {\boldsymbol \alpha}
\newcommand*{\boldbeta}   {\boldsymbol \beta}
\newcommand*{\boldgamma}  {\boldsymbol \gamma}
\newcommand*{\bolddelta}  {\boldsymbol \delta}
\newcommand*{\boldepsilon}{\boldsymbol \epsilon}
\newcommand*{\boldtheta}  {\boldsymbol \theta}

\DeclareMathOperator*{\esssup}{ess\,sup}


\title{\papertitle}
\author{\paperauthor}
\institute{\paperaffil}
\date{\paperdate}

\begin{document}

\AtBeginSection[]{
  \begin{frame}{目次(現在のセクション)}
    % currentsection オプションで,今のセクションだけを強調
    \tableofcontents[
      currentsection,        % 今のセクションを強調
      hideothersubsections   % ほかのセクションのサブセクションは隠す
    ]
  \end{frame}
}
%% Slide 1: タイトル
\begin{frame}
  \titlepage
\end{frame}

\section{導入}
%% Slide 2: オークションとメカニズムの概要
\begin{frame}{本日のアジェンダ}
  \begin{enumerate}
    \item 問題設定 \hfill{\footnotesize}
      \begin{itemize}
        \item 直接メカニズムの定義 
        \item Revelation Principle
      \end{itemize}
    \item IC の含意 \hfill{\footnotesize}
      \begin{itemize}
        \item 接線と\alert{凸関数}性 (式5.5) 
        \item 導関数と\alert{非減少性} (式5.6) 
        \item 積分表示と\alert{ペイオフ等価性} (式5.7) 
      \end{itemize}
    \item 収入等価性\hfill{\footnotesize}
      \begin{itemize}
        \item Proposition 5.2 (式5.8)
        \item Payoff Equivalence の一般化
      \end{itemize}
  \end{enumerate}
\end{frame}

\begin{frame}{5章の概要}
  \begin{itemize}
    \item 売り手は価値の不確実な買い手複数に対し,最適な方式でオブジェクトを配分したい.
    \item オークション形式(入札・配分・支払いルール)以外に,価格ポストや交渉など多様なメカニズムが存在.
    \item 本章では,任意のメカニズムを抽象化し,配分(Allocation)と支払い(Payment)を定式化.
  \end{itemize}
\end{frame}

%=== Preamble に追加 ===
% ブロックを角丸+薄い背景に
\setbeamertemplate{blocks}[rounded][shadow=false]
\setbeamercolor{block title}{fg=white,bg=structure.fg!80!black}
\setbeamercolor{block body}{bg=structure.fg!10!white}
% 列間を詰める
\setlength{\columnsep}{1em}
% itemize の行間を調整
\addtobeamertemplate{itemize/enumerate body begin}{\setlength{\itemsep}{0.5ex}}{}
%======================

% Slide 2: 問題設定と基本定義
\begin{frame}{問題設定と基本定義}
  \begin{columns}[T,onlytextwidth]
    \column{0.48\textwidth}
      \begin{block}{参加者}
        \begin{itemize}
          \item 売り手:非分割オブジェクト1つ(評価額=0)
          \item 買い手 $i\in\mathcal{N}=\{1,\dots,N\}$:リスク中立
          \item 各買い手の真の評価額:$x_i$
        \end{itemize}
      \end{block}

    \column{0.48\textwidth}
      \begin{block}{評価額の分布}
        \begin{itemize}
          \item $x_i\in[0,\omega_i]$ CDF: $F_i$, PDF: $f_i$
          \item 独立性:$f(\bm x)=\prod_{j=1}^N f_j(x_j)$
          \item 他者評価の集合:$\bm x_{-i}$, \\
                $f_{-i}(\bm x_{-i})=\prod_{j\neq i}f_j(x_j)$
        \end{itemize}
      \end{block}
  \end{columns}

  \vspace{1ex}
  \begin{block}{今後の課題}
    配分ルール $\pi$ と支払いルール $\mu$ を定式化し,\\
    最善の配分方法について考える。
  \end{block}
\end{frame}

% Slide: リスク中立性の厳密定義
\begin{frame}{補足:リスク中立}
  \tiny\setlength{\parskip}{0pt}\setlength{\topsep}{0pt}
  \vspace{-1ex}
  \begin{block}{定義(リスク中立)}
    エージェントの効用関数 \(u:\mathbb{R}\to\mathbb{R}\) が,任意のランダム変数 \(W\)(富や支払額)について
    \[
      \mathbb{E}\bigl[u(W)\bigr]
      \;=\;
      u\!\bigl(\mathbb{E}[W]\bigr)
    \]
    を満たすとき,そのエージェントは「リスク中立的」である.
  \end{block}

  \vspace{1ex}
  \begin{block}{同値な条件}
    上記の定義は,次のいずれとも同値である:
    \begin{enumerate}
      \item \(u\) はアフィン関数:  
        \[
          u(w)=a\,w + b,\quad a>0,\;b\in\mathbb{R}.
        \]
      \item 任意の確率分布 \(F_W\) に対して
        \(\displaystyle\int u(w)\,dF_W(w)
        =u\!\bigl(\int w\,dF_W(w)\bigr)\) が成り立つこと.
    \end{enumerate}
  \end{block}

  \vspace{1ex}
  \begin{block}{期待利得の表現}
    リスク中立のもとでは効用=金額とみなせるため,
    買い手 \(i\) の期待利得は
    \[
      \mathbb{E}[\text{評価額}] \;-\; \mathbb{E}[\text{支払額}]
      =\mathbb{E}[X_i] - \mathbb{E}[\mu_i]
    \]
    で表される.
  \end{block}
\end{frame}

\section{5.1}
% Slide 3: メカニズムの定義 (教科書精緻化 1/4)
\begin{frame}{メカニズムの定義}
  売り手が採用するメカニズム $\mathcal{M}=(\mathcal{B},\pi,\mu)$ は次の要素で構成される:
  \begin{description}
    \item[メッセージ集合 $\mathcal{B}=\prod_{i=1}^N\mathcal{B}_i$] 
      各買い手 $i$ が送信できるメッセージ(入札)$b_i\in\mathcal{B}_i$ の集合。
    \item[配分ルール $\pi:\mathcal{B}\to\Delta(\mathcal{N})$] 
      入札ベクトル $b=(b_1,\dots,b_N)$ に対して、
      買い手 $i$ の獲得確率 $\pi_i(b)$ を返す関数。
    \item[支払いルール $\mu:\mathcal{B}\to\mathbb{R}^N$] 
      各買い手 $i$ の期待支払い $\mu_i(b)$ を指定する関数。
  \end{description}
\end{frame}

% Slide: 例: 第1・第2価格オークション(縦積み・詰め詰め版)
\begin{frame}[t]{例: 第1・第2価格オークション}
  \tiny\setlength{\parskip}{0pt}\setlength{\topsep}{0pt}
  \vspace{-1ex}

  \begin{block}{仮定}
    入札域 $\mathcal B_i=\mathcal X_i$ (真値域と同じ)\\
    留保価格(reservation price)は設定しない
  \end{block}
  \vspace{-1ex}

  \begin{block}{配分ルール}
    $\displaystyle
      \pi_i(b)=
      \begin{cases}
        1, & b_i>\max_{j\neq i}b_j,\\
        0, & \text{otherwise},
      \end{cases}
      \quad \sum_{i=1}^N\pi_i(b)=1$
    \\
    {\scriptsize (同点時は均等確率で勝者を選出)}
  \end{block}
  \vspace{-1ex}

  \begin{block}{第一価格オークションの支払い}
    $\displaystyle
      \mu^I_i(b)=
      \begin{cases}
        b_i, & b_i>\max_{j\neq i}b_j,\\
        0, & \text{otherwise}.
      \end{cases}$
  \end{block}
  \vspace{-1ex}

  \begin{block}{第二価格オークションの支払い}
    $\displaystyle
      \mu^{II}_i(b)=
      \begin{cases}
        \max_{j\neq i}b_j, & b_i>\max_{j\neq i}b_j,\\
        0, & \text{otherwise}.
      \end{cases}$
  \end{block}
\end{frame}

% Slide 6: 戦略と均衡の解説
\begin{frame}{戦略と均衡}
  \tiny\setlength{\parskip}{0pt}\setlength{\topsep}{0pt}
  \vspace{-1ex}
  メカニズムは,買い手間に不完全情報下のゲームを定義します。
  \begin{block}{不完全情報とは}
    各買い手 $i$ は自分の真値 $x_i$ のみを知り,\emph{他の買い手の真値は分からない}状況。
    そのため,戦略は自分の $value, type$ 等に依存した関数として定義されます。
  \end{block}

  \vspace{1ex}
  \begin{block}{戦略プロファイル $\beta$}
    各買い手 $i$ は,真値 $x_i\in[0,\omega_i]$ から
    申告(入札)$b_i=\beta_i(x_i)\in\mathcal B_i$ を選択する関数 $\beta_i$ を持つ。
  \end{block}

  \vspace{1ex}
  \textbf{均衡:}\\
  すべての買い手 $i$ とあらゆる真値 $x_i$ について,\emph{他の買い手が
  $\beta_{-i}$ を採用すると仮定したとき},買い手 $i$ の選ぶ申告
  \(\beta_i(x_i)\) は次式で定義される期待利得を最大化します:
  \[
    \beta_i(x_i)
    \;\in\;
    \arg\max_{b_i\in\mathcal{B}_i}
    \underbrace{\mathbb{E}\bigl[\pi_i(b_i,\beta_{-i}(x_{-i}))\cdot x_i
    \;-\;\mu_i(b_i,\beta_{-i}(x_{-i}))\bigr]}_{\text{期待利得}}
  \]
  これが成立する戦略プロファイル $\{\beta_i\}$ を均衡と呼ぶ。
\end{frame}

\begin{frame}{補足:ベイジアンナッシュ均衡について(1/2)}
  不完全情報下の静学ゲームにおいて,プレイヤー \(i\) の利得関数 \(u_i\) は自らのタイプ \(\theta_i\) のみに依存し,
  他のプレイヤーのタイプ集合 \(\theta_{-i}\) には依存しないとする。
  純粋戦略プロファイル \(s^* = (s^*_i)_{i\in I}\) が
  \emph{ベイジアンナッシュ均衡} であるとは,信念集合 \(f = \{f_i\}_{i\in I}\) のもとで
  次が成り立つことを意味します:
  \begin{equation*}
    \begin{aligned}
      &\forall\,i\in I,\;\forall\,\theta_i\in\Theta_i,\;\forall\,a_i\in A_i: \\[-0.5ex]
      &\quad\mathbb{E}_{\theta_{-i}}\bigl[
        u_i\bigl(s^*_i(\theta_i),\,s^*_{-i}(\theta_{-i}),\,\theta_i\bigr)
        \mid \theta_i
      \bigr] \\
      &\quad\ge\;
      \mathbb{E}_{\theta_{-i}}\bigl[
        u_i\bigl(a_i,\,s^*_{-i}(\theta_{-i}),\,\theta_i\bigr)
        \mid \theta_i
      \bigr].
    \end{aligned}
  \end{equation*}
\end{frame}

% Slide: ベイズ・ナッシュ均衡の定義
\begin{frame}[t]{補足:ベイジアンナッシュ均衡 について(2/2)}
  \small
  不完全情報下の静学ゲームにおいて,戦略プロファイル
  \(\beta=(\beta_1,\dots,\beta_N)\) がベイズ・ナッシュ均衡であるとは,
  任意の買い手 \(i\) と任意の真値 \(x_i\) について,
  \[
    \beta_i(x_i)\;\in\;
    \arg\max_{b_i\in\mathcal B_i}
    \;\mathbb{E}_{X_{-i}}\bigl[
      \;\underbrace{\pi_i\bigl(b_i,\beta_{-i}(X_{-i})\bigr)\,x_i}_{\text{獲得確率}\times\text{真値}}
      \;-\;
      \underbrace{\mu_i\bigl(b_i,\beta_{-i}(X_{-i})\bigr)}_{\text{期待支払い}}
    \bigr].
  \]
  ここで \(\mathbb{E}_{X_{-i}}\) は,他の買い手の型 \(X_{-i}\) に関する期待値を表す。
  この条件を全ての \(i,x_i\) で満たす \(\{\beta_i\}\) が BNE である。
\end{frame}

\section{5.1.1}
%% Slide 4: 真実申告メカニズムの導入
\begin{frame}{直接メカニズムの導入}
  \begin{itemize}
    \item 一般メカニズムでは $\mathcal{B}_i$ を任意に設定できるが、単純化のため \textbf{直接メカニズム} では $\mathcal{B}_i=\mathcal{X}_i$ を仮定。
    \item 買い手は評価額をそのまま申告:$b_i=x_i$. こうして得られるメカニズム $(Q,M)$ を
      \emph{直接($direct$)メカニズム} と呼ぶ。
    \begin{align}
  Q &\colon X \to \Delta,\\
  M &\colon X \to \mathbb{R}^N.
    \end{align}
    \item $Q_i(x)$: 評価額ベクトル $x$ のもとで $i$ が獲得する確率
    \item $M_i(x)$: 評価額ベクトル $x$ のもとで $i$ が支払う期待額
  \end{itemize}
\end{frame}

% Slide: 真実申告均衡の定義
\begin{frame}{真実申告均衡の定義}
  \small
  \begin{block}{真実申告均衡(Truthful Equilibrium)}
    直接型メカニズム $(Q,M)$ において,
    各買い手 $i$ の戦略$t_i$を
    \[
      t_i(x_i) \;=\; x_i
    \]
    (自身の真の評価額 $x_i$ をそのまま申告)とする
    戦略プロファイル
    $t=(t_1,\dots,t_N)$ が均衡となる場合,このメカニズムは \emph{真実申告均衡} を有すると言う.
  \end{block}
\end{frame}

%% Slide 6: Revelation原理の命題
\begin{frame}{Revelation原理の命題}
  \begin{block}{命題 5.1 (Revelation Principle)}
    任意のメカニズム $\mathcal{M}=(\mathcal{B},\pi,\mu)$ とその均衡 $\beta$ があれば、
    直接メカニズム $(Q,M)$ を構成して以下を満たす:
    \begin{enumerate}
      \item すべての買い手が真実申告均衡を採用する。
      \item 元のメカニズムの均衡結果(配分・支払い)が再現される。
    \end{enumerate}
  \end{block}
\end{frame}

%% Slide 7: Revelation原理の証明 (1)
\begin{frame}{Revelation原理の証明 (1/2)}
  \textbf{証明の構成:}
  \begin{itemize}
    \item 元メカニズムの均衡戦略 $\beta_i(x_i)$ を用い、
    \item 直接メカニズムの関数 $Q$ と $M$ を次のように定義:
  \end{itemize}

  \begin{center}
    \[
      Q_i(x) \;=\; \pi_i\bigl(\beta(x)\bigr),
      \quad
      M_i(x) \;=\; \mu_i\bigl(\beta(x)\bigr).
    \]
  \end{center}

  この定義から、$Q(x),M(x)$ は $(\pi,\mu)$ と $\beta$ の合成である。
\end{frame}


%% Slide 8: Revelation原理の証明 (2)
\begin{frame}{Revelation原理の証明 (2/2)}
  \begin{itemize}
    \item (1) 真実申告均衡性:
      元メカニズムでは $\beta$ が均衡だったので、不正申告は有利にならない。
      したがって直接メカニズムでも真実申告が最適。
    \item (2) 結果の同一性:
      定義から、任意 $x$ で $Q(x)=\pi(\beta(x))$ 、$M(x)=\mu(\beta(x))$ \newline
      すなわち配分・支払いは元メカニズムと一致。
  \end{itemize}
  \begin{flushright}
    ■
  \end{flushright}
\end{frame}


% --- Slide: Revelation原理の直感 (1/3) ---
\begin{frame}{Revelation原理の直感的理解 (1/3)}
  \begin{block}{ステップ1: 均衡戦略を想定}
    任意のメカニズム $\mathcal{M}=(\pi,\mu)$ とその
    均衡戦略 $\beta=(\beta_1,\dots,\beta_N)$ を固定する。
  \end{block}
  \vspace{1ex}
  ここでは各買い手 $i$ が真値 $x_i$ に応じて
  \[
    b_i = \beta_i(x_i)
  \]
  を提出し,メカニズムにより配分・支払いが決定される状況を想定する。
\end{frame}

% --- Slide: Revelation原理の直感 (2/3) ---
\begin{frame}{Revelation原理の直感的理解 (2/3)}
  \begin{block}{ステップ2: 直接申告メカニズムへの置き換え}
    中間戦略 $\beta_i$ を省略し,\textbf{直接申告}で
    真値 $x_i$ を提出させる。
  \end{block}
  \vspace{1ex}
  直接メカニズムを
  \[
    Q_i(x)\;=\;\pi_i\bigl(\beta(x)\bigr),\qquad 
    M_i(x)\;=\;\mu_i\bigl(\beta(x)\bigr)
  \]
  と定義すれば,元の均衡時の配分・支払いをそのまま再現できる。
\end{frame}

% --- Slide: Revelation原理の直感 (3/3) ---
\begin{frame}{Revelation原理の直感的理解 (3/3)}
  \begin{block}{ステップ3: インセンティブの保持}
    仮にある買い手が真値 $x_i$ の代わりに $z_i$ を申告しても,
    直接メカニズムは「\(\beta_i(z_i)\) を入札した場合」と同じ結果を与える。
  \end{block}
  \vspace{1ex}
  元メカニズムの均衡 $\beta$ では
  \[
    \beta_i(x_i)\;\in\;\arg\max_{b_i}\;
    \mathbb{E}\bigl[\pi_i(b_i,\beta_{-i})\,x_i
      -\mu_i(b_i,\beta_{-i})\bigr],
  \]
  すなわち不正申告 \(z_i\) は有利にならず,直接メカニズムも
  真実申告が均衡を保つ。
\end{frame}

%% Slide 10: Revelation原理の意義と帰結
\begin{frame}{Revelation原理の意義}
  \begin{itemize}
    \item すべてのメカニズム設計問題は「直接メカニズム」の問題に帰着可能。
    \item 設計空間を大幅に狭めても一般性を失わない。
    \item 以降は「真実申告
    メカニズム」に限定して議論を進める。
  \end{itemize}
\end{frame}

\section{5.1.2}

% Slide: 5.1.2 インセンティブ両立性の含意 ― 論理展開
\begin{frame}{5.1.2の概要}
  \small
  本セクションでは,IC条件をまず説明し、次にそれから得られる主な含意を以下の3段階で示します。
  \vspace{1ex}
  \begin{enumerate}
    \item \textbf{凸性の導出}  
      \[
        U_i(x_i)
        =\max_{z_i}\{\,q_i(z_i)x_i - m_i(z_i)\}
        \;\Rightarrow\;
        U_i \text{ は凸関数}
      \]
    \item \textbf{導関数と単調性}  
      :凸関数の性質より
      \[
        U_i'(x_i)=q_i(x_i)
        \quad\Longrightarrow\quad
        q_i \text{ は非減少}
      \]
    \item \textbf{積分表示・ペイオフ等価性}  
      :ほぼ至る所で微分可能かつ,その導関数の積分で元の関数を復元できるので
      \[
        U_i(x_i)=U_i(0)+\int_{0}^{x_i}q_i(t)\,dt,\quad
        m_i(x_i)=m_i(0)+q_i(x_i)x_i-\int_{0}^{x_i}q_i(t)\,dt
      \]
      → 同一配分ルール下で期待収入は定数差のみ
  \end{enumerate}
\end{frame}


% Slide 6a: IC 定義と期待効用
\begin{frame}{インセンティブ両立性条件(Incentive Compatibility)— 設定}
  \begin{itemize}
    \item 直接メカニズム $(Q,M)$ において、申告した値 $z_i$ のときの
    配分確率と支払いを次のように定義する:  
    \begin{equation}\tag{5.1}
      q_i(z_i)
      = \int_{\mathcal{X}_{-i}} Q_i(z_i,x_{-i})\,f_{-i}(x_{-i})\,\mathrm{d}x_{-i},
    \end{equation}
    \begin{equation}\tag{5.2}
      m_i(z_i)
      = \int_{\mathcal{X}_{-i}} M_i(z_i,x_{-i})\,f_{-i}(x_{-i})\,\mathrm{d}x_{-i}.
    \end{equation}
    \small
    \begin{itemize}
      \item $Q_i(z_i,\,x_{-i})$  
        直接メカニズムの配分関数。  
        申告したベクトル $(z_i,\,x_{-i})$ のもとで買い手 $i$ に財を配分する確率
      \item $M_i(z_i,\,x_{-i})$  
        直接メカニズムの支払い関数。  
        申告したベクトル $(z_i,\,x_{-i})$ のもとで買い手 $i$ が支払う金額
      \item $f_{-i}(x_{-i})$  
        他者評価額ベクトル $x_{-i}$ の同時確率密度
    \end{itemize}
    \item 真値 $x_i$ のとき申告 $z_i$ をすると、期待効用は
    \begin{equation}\tag{5.3}
      U_i(x_i;z_i)
      = q_i(z_i)\,x_i \;-\; m_i(z_i).
    \end{equation}
  \end{itemize}
\end{frame}

% Slide 6b: IC 条件
\begin{frame}{IC(インセンティブ両立性条件)について — 定義}
  \begin{block}{インセンティブ両立性条件}
    すなわち、各買い手は真実申告 $z_i=x_i$ を最適化する:
    \begin{equation}\tag{5.4}
      U_i(x_i)\;\equiv\;q_i(x_i)\,x_i - m_i(x_i)
      \;\ge\;
      q_i(z_i)\,x_i - m_i(z_i),
      \quad\forall\,z_i,\,\forall\,x_i.
    \end{equation}
  \end{block}
  \vspace{1ex}
  ここで \alert{$U_i(x_i)$} は\emph{均衡利得関数 (Equilibrium Payoff Function)}と呼ぶ。
\end{frame}


% Slide: IC の含意 
\begin{frame}{IC について — アフィン関数}
  インセンティブ両立性条件の主な含意:
  \begin{enumerate}
    \item 申告した値 $z_i$ を固定すると,期待効用
      \[q_i(z_i)x_i - m_i(z_i)\]
      は真値 $x_i$ に関するアフィン関数となる。
    \item よって
      \[U_i(x_i)=\max_{z_i\in\mathcal{X}_i}\{q_i(z_i)x_i - m_i(z_i)\}\]
      より,$U_i(x_i)$ はアフィン関数族の最大として \textbf{凸関数} である。
  \end{enumerate}
\end{frame}

% Slide: IC の含意 (2/3)
\begin{frame}{IC について— 接線の傾き}
  さらに任意の $x_i,z_i$ に対して,
  \[
    q_i(x_i)z_i - m_i(x_i)
    = U_i(x_i) + q_i(x_i)(z_i - x_i)
  \]
  となるため,IC 条件 (5.4) は
  \[
    U_i(z_i)\ge U_i(x_i) + q_i(x_i)(z_i - x_i)
  \]
  と同値であり,$q_i(x_i)$ は関数 $U_i$ の接線の傾きとなる。
\end{frame}

\begin{frame}{IC について — 接線と凸関数性}
  \begin{equation*}
    \begin{aligned}
      q_i(x_i)\,z_i &- m_i(x_i) \\
      &= \bigl(q_i(x_i)x_i - m_i(x_i)\bigr)
         + q_i(x_i)(z_i - x_i) \\
      &= U_i(x_i) + q_i(x_i)(z_i - x_i).
    \end{aligned}
  \end{equation*}
  よってインセンティブ両立性条件は
  \begin{equation}\tag{5.5}
    U_i(z_i)\;\ge\;U_i(x_i) + q_i(x_i)(z_i - x_i)
  \end{equation}
  と同値となり,\alert{接線}の傾き \(q_i(x_i)\) が示す通り
  \(U_i\) は\alert{凸関数}である。\\[1ex]
\end{frame}

\begin{frame}{IC について — 導関数と非減少性}
  凸関数 $U_i$ は絶対連続かつほとんど至る所で微分可能である。よって
  \begin{equation}
    U_i'(x_i)=q_i(x_i)
    \tag{5.6}
  \end{equation}
  が成り立ち,$q_i$ は\alert{非減少関数}となる。
  \small
  \begin{block}{単調非減少性}
    評価額 \(x_i\) が大きくなるほど,対応する傾き(=割り当て確率)\(q_i(x_i)\) は単調非減少となる:
    \[
      x_i' > x_i
      \quad\Longrightarrow\quad
      q_i(x_i') \;\ge\; q_i(x_i)
      \quad(\forall\,x_i',x_i\in\mathcal{X}_i).
    \]
  \end{block}
\end{frame}

% Slide: 積分表示とペイオフ等価性
\begin{frame}{IC について — 積分表示と{ペイオフ等価性}}
  絶対連続性より,式(5.7) は次のように2行で表せる:
  \begin{equation}\tag{5.7}
    \begin{aligned}
      U_i(x_i)
      &= U_i(0)
       + \int_{0}^{x_i} q_i(t)\,\mathrm{d}t,\\
      m_i(x_i)
      &= m_i(0)
       + q_i(x_i)\,x_i
       - \int_{0}^{x_i} q_i(t)\,\mathrm{d}t.
    \end{aligned}
  \end{equation}
  これにより,期待効用の“形”は配分ルール $q_i$ のみで決まり,
  支払いルール$m_i$ は定数 $U_i(0)$ をシフトするだけである。\\[1ex]
\end{frame}


% Slide: 非減少性がICの必要十分条件
\begin{frame}{IC について ー非減少性との関係}
  IC 条件(5.5)を積分表示(5.7)で書き換えると,
  \[
    \int_{x_i}^{z_i}q_i(t)\,dt \;\ge\; q_i(x_i)(z_i - x_i).
  \]
  この不等式が全ての $x_i<z_i$ で成立するための必要十分条件は,
  \[
    q_i(x_i)\text{ が非減少関数}
  \]
  であること。従って
  \[
    \text{IC}\;\Longleftrightarrow\; q_i\text{ は非減少関数}.
  \]
\end{frame}

% Slide: Revenue Equivalence (Redux) — 命題と等式
\begin{frame}{Revenue Equivalence (Redux)(1/3)}
  \begin{block}{Proposition 5.2 (Revenue Equivalence)}
    直接メカニズム $(Q,M)$ がインセンティブ両立性条件を満たすならば,
    任意の買い手 $i$ と型 $x_i$ に対して期待支払いは
    \[
      m_i(x_i)
      = m_i(0)
      + q_i(x_i)\,x_i
      \;-\;\int_{0}^{x_i}q_i(t)\,dt
      \tag{5.8}
    \]
    で表される。
  \end{block}
  \vspace{1ex}
  すなわち,同一の配分ルール $Q$ を持つ2つの IC メカニズムは,  
  支払いルールが異なっていても,定数差を除いて同一の期待支払い関数を持つ。  
  (Payoff equivalence の一般化)
\end{frame}

% Slide: Revenue Equivalence (Redux) — 証明
\begin{frame}{Revenue Equivalence (Redux) (2/3)}
  教科書より:
  \[
    U_i(x_i) = q_i(x_i)x_i - m_i(x_i),\quad
    U_i(0) = -m_i(0),
  \]
  と定義すると,式(5.7)
  \[
    U_i(x_i)=U_i(0)+\int_{0}^{x_i}q_i(t)\,dt
  \]
  は
  \[
    q_i(x_i)x_i - m_i(x_i)
    = -m_i(0) + \int_{0}^{x_i}q_i(t)\,dt
  \]
  と書き換えられる。整理して
  \[
    m_i(x_i)
    = m_i(0) + q_i(x_i)x_i - \int_{0}^{x_i}q_i(t)\,dt.
  \]
  これが命題 5.2 の主張となる。■
\end{frame}

% Slide: Revenue Equivalence (Redux) — 応用と注意点
\begin{frame}{Revenue Equivalence (Redux)(3/3)}
  \begin{itemize}
    \item $m_i(0)$は実質的に参加料を表す。今回の場合、IR条件のもとで0になる。
    \item 右辺の第3式はIC条件を確保するための情報レントと捉えることができる。
    \item Proposition 5.2 は,Chapter 3 の対称買い手の場合の収入等価性(Proposition 3.1)を拡張したもの。
    \item 買い手が対称かつ効率的配分を行うオークション(例えば第二価格オークション)では,
          $m_i(0)=0$ と定めることで支払いが一意に決まる。
    \item 一般に,\emph{同一の配分ルールを共有する限り},
          すべての IC メカニズムは定数差のみで期待収入が一致する。
  \end{itemize}
\end{frame}

% Slide: 第一価格オークションにおける配分確率と絶対連続性
\begin{frame}{補足:第一価格オークション:配分確率と絶対連続性}
  \small
  \begin{block}{配分確率の定義}
    \[
      q_i(x_i)
      = \int_{\mathcal X_{-i}}
          \mathbf{1}\{\,b(x_i) > \max_{j\neq i} b(x_j)\}
          \prod_{j\neq i} f_j(x_j)\,\mathrm{d}x_{-i}.
    \]
    (同型・対称入札関数 \(b\) の下では)
    \[
      q_i(x_i)
      = \Pr\bigl(X_j \le x_i\;\forall j\neq i\bigr)
      = \bigl[F(x_i)\bigr]^{\,N-1}.
    \]
  \end{block}

  \vspace{1ex}
  \begin{block}{絶対連続性の確認}
    \(F\) は連続分布関数(密度 \(f\) を持つ)なので
    \[
      q_i(x_i)
      = \bigl[F(x_i)\bigr]^{\,N-1}
    \]
    は区間上で微分可能.すなわち
    \[
      q_i'(x_i)
      = (N-1)\bigl[F(x_i)\bigr]^{\,N-2}\,f(x_i)
    \]
    が存在し,
    \(\,q_i(x_i)\) は絶対連続性を満たす.
  \end{block}
\end{frame}


% Slide: 絶対連続性の定義
\begin{frame}{補足:絶対連続性の定義}
  \small
  \begin{block}{直感的定義}
    関数の全変化量は「局所的な変化(傾き)の積み重ね」で説明でき、グラフにジャンプや折れ目が生じない性質。
  \end{block}
  \begin{block}{数学的定義}
    関数 \(f:[a,b]\to\mathbb{R}\) が絶対連続であるとは,次の同値な条件を満たすこと:
    \begin{enumerate}
      \item 任意の \(\varepsilon>0\) に対し,ある \(\delta>0\) が存在して,\newline
        互いに素な有限区間列 \(\{(x_k,y_k)\}\) で \(\sum_k(y_k - x_k)<\delta\) のとき
        \(\sum_k|f(y_k) - f(x_k)|<\varepsilon\) が成り立つ.
      \item \(f\) はほとんど至るところで微分可能かつ \(f'\) が可積分であり,
        \[
          f(x) \;=\; f(a) \;+\; \int_a^x f'(t)\,\mathrm{d}t
          \quad(\forall\,x\in[a,b]).
        \]
    \end{enumerate}
  \end{block}
\end{frame}


\section{補足:アフィン関数と凸関数}
% Slide: 凸性の理解—Affine 関数族の最大
\begin{frame}{補足:凸性の理解—Affine 関数族の最大 (1/3)}
  \begin{block}{命題}
    任意の集合 $Z$ 上の affine 関数族
    \[
      \{\,f_z(x) = a_z x + b_z \mid z\in Z,\;a_z,b_z\in\mathbb R\}
    \]
    に対し,
    \[
      U(x) \;=\;\max_{z\in Z}\{\,f_z(x)\}
    \]
    と定義すると,$U(x)$ は凸関数である。
  \end{block}
  \vspace{1ex}
  \textbf{直感:}
  Affine 関数はすべて傾き一定の直線。これらの「上包絡線」を取ると、  
  任意の区間で「線分」が常に関数の上を覆うため、凸性が保証される。
\end{frame}

% Slide: 凸性の証明—包絡線定理との関係
\begin{frame}{補足:凸性の証明—包絡線定理との関係 (2/3)}
  \begin{proof}
    任意の $x_1,x_2\in\mathbb R$ と $\lambda\in[0,1]$ を取る。
    \[
      U\bigl(\lambda x_1 + (1-\lambda)x_2\bigr)
      = \max_{z}\{\,a_z(\lambda x_1+(1-\lambda)x_2)+b_z\}.
    \]
    ここで、各 $z$ について
    \[
      a_z(\lambda x_1+(1-\lambda)x_2)+b_z
      = \lambda\,(a_z x_1+b_z)
      + (1-\lambda)\,(a_z x_2+b_z).
    \]
    よって
    \[
      U(\lambda x_1+(1-\lambda)x_2)
      \le \max_z \bigl[\lambda f_z(x_1)+(1-\lambda)f_z(x_2)\bigr]
      \le \lambda\,U(x_1)+(1-\lambda)\,U(x_2).
    \]
    これで凸性 $U(\lambda x_1+(1-\lambda)x_2)\le\lambda U(x_1)+(1-\lambda)U(x_2)$ を示した。
  \end{proof}
\end{frame}

% Slide: 経済学への応用—Envelope 定理とのリンク
\begin{frame}{補足:経済学への応用—Envelope 定理とのリンク (3/3)}
  \begin{itemize}
    \item メカニズム設計では,
      \(\displaystyle U_i(x_i)=\max_{z_i}\{\,q_i(z_i)x_i - m_i(z_i)\}\)
      の形で凸性を得る。
    \item Envelope 定理を使うと,最適 \(z_i(x_i)\) に関して
      \[
        \frac{dU_i}{dx_i} = q_i\bigl(z_i(x_i)\bigr)
        \;\approx\; q_i(x_i)
      \]
      が成り立ち,IC 条件の微分形につながる。
    \item 結論: 凸性と包絡線の構造が,IC 条件や支払い関数の一意性論証の基盤。
  \end{itemize}
\end{frame}

% --- Slide: Envelope 定理の数学的形態詳細 (1/2) ---
\begin{frame}{補足:Envelope 定理の数学的形態詳細 (1/2)}
  \begin{block}{一般的な Envelope 定理}
    関数 \(f(x,z)\) が
    \begin{itemize}
      \item \(x\) について連続的に微分可能、
      \item \(z\) についてコンパクト集合 \(\mathcal Z\) 上で定義、
      \item \(f\) と最適選択 \(z^*(x)\) が適切に可測
    \end{itemize}
    を満たすとき、
    \[
      V(x)=\max_{z\in\mathcal Z}f(x,z)
      \quad\Longrightarrow\quad
      V'(x)=\frac{\partial f}{\partial x}\bigl(x,z^*(x)\bigr).
    \]
  \end{block}
  \vspace{1ex}
  メカニズム設計では
  \(\;f(z_i;x_i)=q_i(z_i)\,x_i - m_i(z_i)\;\)
  をこの形に当てはめます。
\end{frame}


% --- Slide: Envelope と Affine 関数族の上包絡線 (2/2) ---
\begin{frame}{補足:Envelope と Affine 関数族の上包絡線 (2/2)}
  \begin{itemize}
    \item \(f(z;x)=q(z)x - m(z)\) は \(x\) に関する affine(線形+定数項)の族を構成。
    \item その上包絡線(upper envelope)とは、点ごとに「最大値を与える直線」を継ぎ合わせた曲線であり、
      \[
        U(x)=\max_z\{\,q(z)x - m(z)\,\}
      \]
      と一致する。この関数は凸性を持つ。
    \item 幾何的に、上包絡線の任意点での接線の傾きが
      \(\displaystyle U'(x)=q\bigl(z^*(x)\bigr)\)
      に対応する。
  \end{itemize}
  \vspace{1ex}
  この2つの視点(Envelope 定理 vs. 上包絡線)が合致することで、
  インセンティブ互換性(IC)の微分形
  \(\;U_i'(x_i)=q_i(x_i)\;\)
  が厳密に導かれます。
\end{frame}

\section{次回:5.1.3}

% Slide: 個人合理性 (Individual Rationality) 1/3
\begin{frame}{(次回)個人合理性 (Individual Rationality)}
  \begin{block}{定義}
    直接メカニズム $(Q,M)$ は \emph{個人合理的 (individually rational)} であるとは,
    \[
  \forall\,i\in N,\quad \forall\,x_i\in X_i,\quad U_i(x_i)\;\ge\;0.
    \]
    つまり、
    \[
      U_i(x_i)\equiv q_i(x_i)\,x_i \; - \; m_i(x_i)\;\ge\;0
    \]
    が成り立つことである。
  \end{block}
  \vspace{1ex}
  ここで非参加時の利得を $0$ と仮定し,\emph{参加しない選択肢}を保証する条件としている。
\end{frame}

% Slide: IC 下での個人合理性の帰結
\begin{frame}{(次回)IC 下での個人合理性の帰結}
  \begin{itemize}
    \item IC より,式(5.7)
      \[U_i(x_i)=U_i(0)+\int_{0}^{x_i}q_i(t)\,dt\]
      が成り立つ。
    \item よって $U_i(x_i)\ge0$ は最も厳しく
      \[U_i(0)\ge0\]
      を要求する。
    \item 定義 $U_i(x_i)\equiv q_i(x_i)x_i - m_i(x_i)$ から
      \[U_i(0)=-m_i(0)\]
    \item 結論:
      \[U_i(0)\ge0\iff m_i(0)\le0\]
      すなわち $x_i=0$ のとき支払いを徴収しない。
  \end{itemize}
\end{frame}

% Slide: 個人合理性の解釈と応用例 3/3
\begin{frame}{(次回)個人合理性の解釈と応用例}
  \begin{itemize}
    \item $m_i(0)\le0$ は,「価値 0 の参加者が負担を負わない」ことを意味。
    \item 通常は基準化として $m_i(0)=0$ を採用し,\emph{参加費なし}を保証。
    \item 個人合理性を満たすことで,\emph{全ての買い手が少なくとも不参加時より良いか同等}となり,
      メカニズムへの参加を確実に誘導できる。
    \item reserve価格設定や補助金付きオークション設計の際に関係する
  \end{itemize}
\end{frame}

\end{document}


\section{Appendix}
\begin{frame}{A.2 RegretNet:組合せ評価への対応}
  \small
  \begin{itemize}
    \item 本節では、一般的な組合せ型評価(combinatorial valuations)を持つ入札者に対応するために、RegretNetアーキテクチャをどのように調整するかを示す.
    \item この設定では、各入札者 $i$ は、空集合を除くすべてのアイテム集合 $S \subseteq M$ に対して、ビッド $b_{i,S}$ を報告する(空集合に対する評価は 0 とする)。
    \item ネットワークの割当コンポーネント(allocation component)は、各入札者 $i$ と各バンドル $S$ に対して出力 $z_{i,S} \in [0, 1]$ を持つ。
    \item これは、入札者 $i$ にバンドル $S$ が割り当てられる確率を意味する。
  \end{itemize}
\end{frame}

\begin{frame}{A.2 RegretNet:組合せ割当のconstraintとスコア設計}
  \small
  \begin{itemize}
    \item アイテムの過剰割当を防ぐため、次の2つのconstraintを課す:
    \begin{enumerate}
      \item 各アイテム $j \in M$ に対し、それが割当てられる確率の合計は高々1:
      \[
      \sum_{i \in N} \sum_{S \subseteq M: j \in S} z_{i,S} \leq 1 \quad \text{(式14)}
      \]
      \item 各入札者 $i \in N$ に対し、バンドルの割当確率の合計も高々1:
      \[
      \sum_{S \subseteq M} z_{i,S} \leq 1 \quad \text{(式15)}
      \]
    \end{enumerate}
    \item これらのconstraintを満たす割当を「組合せ的に実現可能(combinatorial feasible)」と呼ぶ。
    \item constraintを実装するため、ネットワークの割当コンポーネントは以下のA group of scoresを計算する:
    
    \begin{itemize}
      \item 各入札者 $i \in N$ に対して:すべてのバンドル $S \subseteq M$ にスコア $s_{i,S}$
      \item 各アイテム $j \in M$ に対して:すべての $i \in N$ と $S \subseteq M$ にスコア $s^{(j)}_{i,S}$
    \end{itemize}
  \end{itemize}
\end{frame}

\begin{frame}{A.2 RegretNet:割当スコアの正規化と定義}
  \small
  \begin{itemize}
    \item A group of scoresを以下のように定義する:
    \begin{itemize}
      \item 入札者ごとのscores in matrix form:$s \in \mathbb{R}^{n \times 2^m}$
      \item アイテムごとのA group of scores:$s^{(1)}, \ldots, s^{(m)} \in \mathbb{R}^{n \times 2^m}$
    \end{itemize}

    \item Each group of scoresはソフトマックス関数によって正規化される:
    \[
    \bar{s}_{i,S} = \frac{\exp(s_{i,S})}{\sum_{S'} \exp(s_{i,S'})},       \quad
    \bar{s}^{(j)}_{i,S} = \frac{\exp(s^{(j)}_{i,S})}{\sum_{i', S'} \exp(s^{(j)}_{i', S'})}
    \]

    \item 各入札者 $i$ に対するバンドル $S \subseteq M$ の割当確率は以下の最小値として定義される:
    \[
    z_{i,S} = \phi^{\text{CF}}_{i,S}(s, s^{(1)}, \ldots, s^{(m)}) = \min \left( \bar{s}_{i,S}, \, \bar{s}^{(j)}_{i,S} \text{ for all } j \in S \right)
    \]
  \end{itemize}
\end{frame}

\begin{frame}{A.2 実現可能性}
  \small
  \textbf{定義 A.1:実現可能性(Implementability)} \\
  Fractionalな組合せ割当 $z$ が実現可能であるとは、それが \textbf{組合せ的に実現可能な確定的割当}の凸結合として表現できる場合をいう。

  \vspace{0.5em} 
  しかし、以下の例により、ある組合せ的に実現可能な割当でも、整数的な分解(integer decomposition)が存在しないことが示される。
\end{frame}

\begin{frame}{A.2 非実現可能な例}
  \small
  \textbf{例 A.1:実現不能な組合せ割当} \\
  2人の入札者と2つのアイテムがある設定において、次のようなFractionalな割当を考える:
  \[
  z = \begin{bmatrix}
  z_{1,\{1\}} & z_{1,\{2\}} & z_{1,\{1,2\}} \\
  z_{2,\{1\}} & z_{2,\{2\}} & z_{2,\{1,2\}}
  \end{bmatrix}
  = \begin{bmatrix}
  \frac{3}{8} & \frac{3}{8} & \frac{1}{4} \\
  \frac{1}{8} & \frac{1}{8} & \frac{1}{4}
  \end{bmatrix}
  \]

  この $z$ を次のようなInteger Allcationの凸結合で表そうとすると:

  \[
  z = 
  a \begin{bmatrix} 0 & 0 & 1 \\ 0 & 0 & 0 \end{bmatrix}
  + b \begin{bmatrix} 0 & 0 & 0 \\ 0 & 0 & 1 \end{bmatrix}
  + c \begin{bmatrix} 1 & 0 & 0 \\ 0 & 1 & 0 \end{bmatrix}
  + d \begin{bmatrix} 1 & 0 & 0 \\ 0 & 0 & 0 \end{bmatrix}
  + e \begin{bmatrix} 0 & 0 & 0 \\ 0 & 1 & 0 \end{bmatrix}
  \]
  \[
  + f \begin{bmatrix} 0 & 1 & 0 \\ 1 & 0 & 0 \end{bmatrix}
  + g \begin{bmatrix} 0 & 1 & 0 \\ 0 & 0 & 0 \end{bmatrix}
  + h \begin{bmatrix} 0 & 0 & 0 \\ 1 & 0 & 0 \end{bmatrix}
  \]


  係数は $a = b = \frac{1}{4}$、$c + d = \frac{3}{8}$、$f + g = \frac{3}  {8}$ を満たす必要がある。

  しかし合計は:
  \[
  a + b + c + d + e + f + g + h \geq \frac{1}{2} + \frac{3}{4} = \frac{5}{4}
  \]
  となり、1 を超えるため、$z$ は実現可能ではない。
\end{frame}

\begin{frame}{A.2 Integer Decompositionを保証するconstraintと定理A.3}
  \small
  \begin{itemize}
    \item 組合せ的に実現可能な割当 $z$ がinteger decomposition possibleであることを保証するには、追加のconstraintが必要。
    \item 2アイテム($m=2$)の場合、次のconstraintを導入する:
    \[
    \forall i, \quad z_{i,\{1\}} + z_{i,\{2\}} \leq 1 - \sum_{i'=1}^{n} z_{i',\{1,2\}} \quad \text{(式17)}
    \]
    \item このconstraintは、2つのアイテムが両方とも含まれるバンドルの過剰割当を抑制する。
  \end{itemize}
\end{frame}

\begin{frame}{A.2 Integer Compositionを保証するconstraintと定理A.3}
  \small
  \textbf{定理 A.3}:\\
  $m = 2$ のとき、additional constraint(式17)を満たす任意の組合せ的に実現可能な割当 $z$ は、以下のように表すことができる:
  \[
  z = \sum_{k} \lambda_k B_k
  \] 
  ここで、各 $B_k$ は 0-1 行列で、組合せ的に実現可能な割当である(すなわち、確定的な割当)。\\
  \vspace{0.5em}
  \textbf{方針}:$z$ を、{1,2}バンドルを割り当てる部分 $B_i$ と、それ以外の部分 $C$ に分解する。\\
  \vspace{0.5em}
  \textbf{観察}:確定的な割当 $B$ において、ある $i$ に対して $B_{i,\{1,2\}} = 1$ であれば、他のすべての入札者 $j \ne i$ に対し、$B_{j,S} = 0$ でなければならない。
\end{frame}


\begin{frame}{A.2 定理A.3の証明(m=2のInteger Composition Possible)}
  \small
  \textbf{ステップ1:分解}
  \[
  z = \sum_{i=1}^{n} z_{i,\{1,2\}} \cdot B_i + C
  \]
  ここで、
  \[
  B_i(j,S) = 
  \begin{cases}
  1 & \text{if } j = i,\, S = \{1,2\} \\
  0 & \text{otherwise}
  \end{cases}
  \]

  \vspace{0.5em}
  \textbf{ステップ2:残差部分 $C$ の性質}
  \begin{itemize}
    \item $C$ の {1,2} 列はすべて 0。
    \item 各アイテムについて:
      \[
      \sum_i C_{i,\{1\}} \leq 1 - \sum_i z_{i,\{1,2\}}, \quad
      \sum_i C_{i,\{2\}} \leq 1 - \sum_i z_{i,\{1,2\}}
      \]
    \item さらに、additional constraint(17)より:
      \[
      C_{i,\{1\}} + C_{i,\{2\}} = z_{i,\{1\}} + z_{i,\{2\}} \leq 1 - \sum_i z_{i,\{1,2\}}
      \]
  \end{itemize}
\end{frame}

\begin{frame}{A.2 定理A.3の結論}
  \textbf{結果}:$C$ はScaling Factor  $1 - \sum_i z_{i,\{1,2\}}$ を持つ二重確率行列(doubly stochastic matrix)であり、したがって $C$ は可行な0-1割当行列の線形結合として分解可能。
  \[
  C = \sum_k p_k \cdot B_k, \quad \sum_k p_k \leq 1 - \sum_i z_{i,\{1,2\}}
  \]
\end{frame}

\begin{frame}{A.2.1 RegretNet:2アイテムオークションへの拡張}
  \small
  \textbf{目的}:2アイテムのケースで、additional constraint(式17)を満たすようにするため、各入札者に対して追加のSoftmax層を導入する。

  \vspace{0.5em}
  \textbf{従来の input scores(入札者・アイテムごと)}:
  \begin{itemize}
    \item un-normalized bidder-wise scores:$s_{i,S}$($\forall i \in N$, $S \subseteq M$)
    \item item-wise score:$s^{(j)}_{i,S}$($\forall i \in N$, $S \subseteq M$, $j \in M$)
    \item それぞれに対応するSoftmax-normalized score:$\bar{s}_{i,S}$ および $\bar{s}^{(j)}_{i,S}$
  \end{itemize}

  \vspace{0.5em}
  \textbf{Implementability対応の追加スコア(補助Softmax)}:
  \begin{itemize}
    \item 各入札者 $i$ に対し、以下のAdditional Scores $s^{(i)}_{k,S}$ を計算する:\\
     $s^{(i)}_{i,\{1\}}$, $s^{(i)}_{i,\{2\}}$, $s^{(i)}_{1,\{1,2\}}, \ldots, s^{(i)}_{n,\{1,2\}}$
\[
\begin{aligned}
&\text{for all } i, k \in N,\ S \subseteq M, \\
&\quad \bar{s}^{(i)}_{k,S} = 
  \frac{\exp(s^{(i)}_{k,S})}{
    \exp(s^{(i)}_{i,\{1\}}) + \exp(s^{(i)}_{i,\{2\}}) +
    \sum_{k'} \exp(s^{(i)}_{k',\{1,2\}})
  }
\end{aligned}
\]

  \end{itemize}
\end{frame}


\begin{frame}{A.2.1 最終割当と支払いの定義}
  \small
  \textbf{Additional constraint(17)を満たすために補正したnormalized score $\bar{s}'_{i,S}$ の定義}:

  \[
  \bar{s}'_{i,S} =
  \begin{cases}
  \bar{s}^{(i)}_{i,S} & \text{if } S = \{1\} \text{ or } \{2\} \\
  \min\left\{ \bar{s}^{(k)}_{i,S} : k \in N \right\} & \text{if } S = \{1, 2\}
  \end{cases}
  \]

  \vspace{0.5em}
  \textbf{最終的な割当確率 $z_{i,S}$ の定義}:
  \[
  z_{i,S} = \min\left( \bar{s}_{i,S}, \, \bar{s}'_{i,S}, \, \bar{s}^{(j)}_{i,S} : j \in S \right)
  \]

  \vspace{0.5em}
  \textbf{支払いコンポーネント}(Figure 3と同様の構造):
  \begin{itemize}
    \item 各入札者 $i$ に対して、Sigmoid Unitを用いてFractional Payment $\tilde{p}_i \in [0,1]$ を計算。
    \item 最終的な支払い額:
    \[
    p_i = \tilde{p}_i \cdot \sum_{S \subseteq M} z_{i,S} \cdot b_{i,S}
    \]
  \end{itemize}
\end{frame}
