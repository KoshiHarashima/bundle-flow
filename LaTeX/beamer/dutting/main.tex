\documentclass[dvipdfmx,autodetect-engine]{beamer}

% 好みのテーマ(必要なら後で変更)
\usetheme{Madrid}
\usecolortheme{default}

% 共通前置き&マクロ
% ==== beamer 用(upLaTeX + dvipdfmx 前提)====

% 数式
\usepackage{amsmath,amssymb,mathtools,bm}
\allowdisplaybreaks

% 図(beamerはgraphicxを内部導入するが,dvipdfmx用途で明示してもよい)
\usepackage[dvipdfmx]{graphicx}

% ハイパーリンク:beamerがhyperrefを読むので,細部設定のみ
\hypersetup{
  hidelinks,
  pdfpagemode=UseNone,
  bookmarksnumbered=true
}

% ---- 文献(biblatex + biber 前提;Paperpileの .bib を主ファイル基準で)----
\usepackage{csquotes}
\usepackage[
  backend=biber,
  style=apa,      % 必要に応じて phys/authoryear へ差替え
  natbib=true
]{biblatex}
\addbibresource{paperpile.bib}

% 参照(定理名や図表名のスマート参照)
\usepackage[nameinlink,capitalise]{cleveref}

% 表(論文向けの見栄え)
\usepackage{booktabs,threeparttable,threeparttablex}
\usepackage{tabularx,array}
\renewcommand{\arraystretch}{1.1}

% 色
\usepackage{xcolor}

% 日本語(upLaTeX)
\usepackage[deluxe]{otf}
\usepackage[noalphabet]{pxchfon}
\renewcommand{\kanjifamilydefault}{\gtdefault}
\IfFileExists{HaranoAjiGothic-Medium.otf}{%
  \setboldgothicfont{HaranoAjiGothic-Medium.otf}%
}{}

% 図の相対パス
\graphicspath{{./figures/}}




% ==== メタ情報(必要に応じて毎回ここだけ更新)====
\newcommand{\papertitle}{N}
\newcommand{\paperauthor}{Koshi Harashima}
\newcommand{\paperaffil}{Northwestern University}
\newcommand{\paperdate}{October 6, 2025}

% ==== 数式系のマクロ ====
\newcommand*{\N}{\mathbb{N}}
\newcommand*{\Z}{\mathbb{Z}}
\newcommand*{\Q}{\mathbb{Q}}
\newcommand*{\R}{\mathbb{R}}
\newcommand*{\C}{\mathbb{C}}
\newcommand*{\I}{\mathbb{I}}

\newcommand*{\boldalpha}  {\boldsymbol \alpha}
\newcommand*{\boldbeta}   {\boldsymbol \beta}
\newcommand*{\boldgamma}  {\boldsymbol \gamma}
\newcommand*{\bolddelta}  {\boldsymbol \delta}
\newcommand*{\boldepsilon}{\boldsymbol \epsilon}
\newcommand*{\boldtheta}  {\boldsymbol \theta}

\DeclareMathOperator*{\esssup}{ess\,sup}


\title{\papertitle}
\author{\paperauthor}
\institute{\paperaffil}
\date{\paperdate}

\begin{document}

\begin{frame}
  \titlepage
\end{frame}

\begin{frame}{Abstract}
  \small
  \begin{itemize}
    \item 発表の趣旨:
    \begin{itemize}
      \item 複数アイテムに対するICかつ収入が最大となるオークション設計について考える.
      \item Krishnaと問題設定や表記が異なるため整理する.
      \item 確率的な割り当てについての考察を加えた。
      \item ベクトルは太字で表記した。
    \end{itemize}
\end{itemize}
\end{frame}

\section{2-1}
\begin{frame}{2.1  Preliminaries:評価関数}
  \small
  \begin{itemize}
    \item 参加者 ($bidders$) の集合 $N = \{1, \ldots, n\}$ と,\\
          アイテム ($items$) の集合 $M = \{1, \ldots, m\}$ を考える.\\
          どちらも有限の集合である.
    {\small
    \item 各参加者 $i$ は評価関数($Valuation Function$)  
      \[
        v_i: 2^M \;\rightarrow\;\mathbb{R}_{\ge0}
      \]
      を持つ.
      特に確率的な割り当てに関しては、
      \[
        v_i: \mathbb{R}^{2^{|M|}} \;\rightarrow\;\mathbb{R}_{\ge0}
      \]
      が成り立つ。これは任意の部分集合 $S \subseteq M$ に対し,$v_i(S)$ がその価値を表す.空集合なら評価は0とみなす.
    \item [補足1]$2^M$ は,集合 $M$ の すべての部分集合を集めた集合(冪集合)を表す.すなわちアイテムが $m$ 個あるとき,その個数(濃度)は $|2^M|= 2^{\lvert M \rvert} = 2^m$ .\\
    たとえば,
    \[
    M=\{A,B\}
    \quad\text{ならば}\quad
    2^{M}= \bigl\{\,\varnothing,\;\{A\},\;\{B\},\;\{A,B\}\bigr\}.
    \]
    \item [補足2]部分集合はbundleに他ならない.}
  \end{itemize}
\end{frame}


\begin{frame}{2.1  Preliminaries:評価関数の例}
  \small
  $2^{\lvert M \rvert}$について全ての評価値を毎回計算すると大変である.以下の例では個別の評価値のみを考えるため、$2^{\lvert M \rvert}$を考える代わりに,|M|個で済む。\\
  代表的な評価関数($Valuation Function$)の例:
  \begin{itemize}
    \item 加法型($Additive$):  \\
    各アイテム $j \in M$ に個別の評価 $v_i(\{j\})$ をそれぞれ持ち,任意の部分集合 $S \subseteq M$ に対して:  
    \[
    v_i(S) = \sum_{j \in S} v_i(\{j\})
    \]
    \item ユニットデマンド型($Unit-Demand$):\\  
    部分集合 $S \subseteq M$ に対して,最も価値の高いアイテム1つの価値を取り,それを部分集合の評価とする.:  
    \[
    v_i(S) = \max_{j \in S} v_i(\{j\})
    \]
    \item 一般の組合せ型($General Combinatorial$):\\  
    詳細はAppendix A.2 および B.3 に記載.
  \end{itemize}
\end{frame}

\begin{frame}{2.1  Preliminaries:オークションの設定}
  \small
  \begin{itemize}
    \item 参加者 $i$ に取り得る評価関数の集合を $\mathcal{V}_i$ とする.  
          その直積
          \[
            \mathcal{V}\;=\;\mathcal{V}_1 \times \cdots \times \mathcal{V}_n
          \]
          が「評価関数が取りうる空間」である.\\
          各 $i$ の真の評価関数 $v_i$ は分布 $F_i$( $\mathcal{V}_i$上で定義される)から独立に抽出される.その組を $\mathbf{v}=(v_1,\ldots,v_n)$ と書く.
    \item 主催者(auctioneer)は分布  
          $\mathbf{F}=(F_1,\ldots,F_n)$ を既知とするが,実現した評価の組 $\mathbf{v}$ は観測できない.
    \item 各参加者は(真実とは限らない)入札 $\mathbf{b}=(b_1,\ldots,b_n)$ を提出し,  
          メカニズム $(g,p)$ が入札に基づいて  
          割当 $g(\mathbf{b})\in(2^{M})^{n}$ と支払い $p(\mathbf{b})\in\mathbb{R}^{n}_{\ge0}$ を決定する.
  \end{itemize}
\end{frame}

\begin{frame}{2.1  Preliminaries:メカニズム}
  \small
  \begin{itemize}
    \item オークションは $(g, p)$ のペアとして表される.
     \begin{itemize}
        \item 割り当て($Allocation$) $g_i: \mathcal{V} \rightarrow 2^M$
        \item 確率的な割り当て($Allocation$) $g_i : \mathcal{V}\;\longrightarrow\mathcal{P}$,
        \begin{equation*}
        \textstyle
        \mathcal{P}=\bigl\{\,
        q\in\mathbb{R}^{2^{|M|}}_{\ge0}\;:\;
        \forall m\in M,\;
        0\leq q(S)\le1
        \bigr\}.
        \end{equation*}
        \item 支払い($Payment$) $p_i: \mathcal{V} \rightarrow \mathbb{R}_{\geq 0}$
      \end{itemize}
    \item オークションでの入力と出力について:
    \begin{itemize}
        \item 各プレイヤー $i$ は評価関数
        $b_i:2^{M}\to\mathbb{R}_{\ge 0}$ を申告する(自己申告メカニズム).
        \item 入力として,入札 $\mathbf{b}\ = (b_1, \ldots, b_n) \in \mathcal{V}$ に対して,
        \item 以下の式のように割り当てと支払いを出力する(確率的割り当てについては次ページ).
        \begin{equation*}
        \begin{aligned}
        g(\mathbf{b}) &= \bigl(g_{1}(\mathbf{b}),\dots,g_{n}(\mathbf{b})\bigr)
                      \;\in\;(2^{M})^{\,n}\\[4pt]
        p(\mathbf{b}) &= \bigl(p_{1}(\mathbf{b}),\dots,p_{n}(\mathbf{b})\bigr)
                      \;\in\;\mathbb{R}^{\,n}_{\ge 0}.
        \end{aligned}
        \end{equation*}
    \end{itemize}
  \end{itemize}
\end{frame}

\begin{frame}{2.1  Preliminaries:確率的な割り当て}
%--- 束(部分集合)の命名 ----------------------------------------
\newcommand{\numS}{2^{|M|}}        % 束の総数
\newcommand{\Sk}{S_k}              % 束 S_k
\newcommand{\Sik}{S_{i,k}}         % プレイヤー i に対応づけた S_k
\newcommand{\qik}{q_{i,k}(\mathbf{b})} % その確率

%--- 割当ベクトルの定義 ------------------------------------------
\[
  g(\mathbf{b})
  \;=\;
  \bigl(
    g_1(\mathbf{b}),
    \dots,
    g_n(\mathbf{b})
  \bigr),
  \qquad
  g_i(\mathbf{b})
  \;=\;
  \bigl(
    \qik
  \bigr)_{k=1}^{\numS}
  \;\in\;
  [0,1]^{\numS}.
\]


\[
  \boxed{
  \begin{aligned}
    &\text{(1) item-wise feasibility} &
      &\sum_{i=1}^{n}\sum_{k:\,m\in\Sk} \qik \;\le\; 1
      &&\forall\,m\in M, \\[6pt]
    &\text{(2) probability} &
      &0 \;\le\; \qik \;\le\; 1
      &&\forall\,i,\;k.
  \end{aligned}
  }
\]
\end{frame}


\begin{frame}{2.1  Preliminaries:効用}
  \small
  \begin{itemize}
    \item 評価関数 $v_i$ を持つ参加者の効用は,入札の組\(\mathbf{b}\)において次式で与えられる:
    \[
    u_i(v_i;\mathbf{b}\ ) = v_i(g_i(\mathbf{b}\ )) - p_i(\mathbf{b}\ )
    \]
    \item 以下の略記をこれまでと同じように採用する:
    \begin{align*}
    \mathbf{v}_{-i}           &= (v_1,\ldots,v_{i-1},v_{i+1},\ldots,v_n),\\
    \mathbf{b}_{-i}           &= (b_1,\ldots,b_{i-1},b_{i+1},\ldots,b_n),\\
    \mathcal{V}_{-i} &= \prod_{j\neq i}\mathcal{V}_j.
\end{align*}
  \end{itemize}
\end{frame}

\begin{frame}{2.1  Preliminaries:DSICとIR}
  \small
  Krishnaでは,BICとInterim IRで考察をしたが,\\
  DüttingではDSICとEx-post IRで考察をする.\\
  \textbf{Dominant Strategy Incentive Compatible (DSIC)}:
  \begin{itemize}
    \item 他の参加者のビッドに関係なく,真実の申告が常に最善である場合,そのオークションはDSICである.
    \begin{equation*}
    \begin{aligned}
    &\forall\,i,\;\forall\,v_i\in\mathcal{V}_i,\;
      \forall\,b_i\in\mathcal{V}_i,\;
      \forall\,\mathbf{b}_{-i}\in\mathcal{V}_{-i}:\\[2pt]
    &\qquad
      u_i\!\bigl(v_i;\,(v_i,\mathbf{b}_{-i})\bigr)
      \;\ge\;
      u_i\!\bigl(v_i;\,(b_i,\mathbf{b}_{-i})\bigr).
    \end{aligned}
    \end{equation*}
  \end{itemize}

  \vspace{0.5em}
  \textbf{Ex\textendash{}post Individually Rational (IR)}:
  \begin{itemize}
    \item 参加者が真実を申告した場合,効用が常に0以上であるとき,オークションはIRである.
    \begin{equation*}
    \begin{aligned}
    &\forall\,i,\;\forall\,v_i\in\mathcal{V}_i,\;
      \forall\,b_i\in\mathcal{V}_i,\;
      \forall\,\mathbf{b}_{-i}\in\mathcal{V}_{-i}:\\[2pt]
    &\qquad
      u_i\!\bigl(v_i;\,(v_i,\mathbf{b}_{-i})\bigr)\;\ge\;0.
    \end{aligned}
    \end{equation*}
  \end{itemize}
\end{frame}

\begin{frame}{前回までの復習:ICについて}
\small
\textbf{DSIC:Dominant Strategy Incentive Compatibility}

\begin{itemize}
  \item 各参加者が,他の参加者の行動に関係なく,真実を申告するのが常に最適な戦略である場合.
\begin{align*}
  &\forall\,i,\;\forall\,v_i,b_i\in\mathcal{V}_i,\;
    \forall\,\mathbf{b}_{-i}\in\mathcal{V}_{-i}:\\[2pt]
  &\quad
    u_i\!\bigl(v_i;\,(v_i,\mathbf{b}_{-i})\bigr)
    \;\ge\;
    u_i\!\bigl(v_i;\,(b_i,\mathbf{b}_{-i})\bigr).
\end{align*}
\end{itemize}

\vspace{1em}

\textbf{BIC:Bayesian Incentive Compatibility}

\begin{itemize}
  \item 各参加者が,他の参加者も真実を申告するという前提の下で,真実を申告するのが最善である場合.
\begin{align*}
  \forall\,i,\;\forall\,v_i\in\mathcal{V}_i,b_i\in\mathcal{V}_i:\quad\\[2pt]
  \mathbb{E}_{\mathcal{V}_{-i}}\!\Bigl[\,u_i\!\bigl(v_i;\,(v_i,\mathbf{v}_{-i})\bigr)\Bigr]
  &\ge
  \mathbb{E}_{\mathcal{V}_{-i}}\!\Bigl[\,u_i\!\bigl(v_i;\,(b_i,\mathbf{v}_{-i})\bigr)\Bigr].
\end{align*}
\end{itemize}
\end{frame}

\begin{frame}{前回までの復習:IRについて}
  \small
  \begin{enumerate}
  \item \textbf{Ex\textendash{}ante Individually Rational}:\\
   各プレイヤー \(i\)が,typeが実現する前の参加を決める段階で,すべての参加者において,真実申告をしたときの、期待効用が非負である場合.
  \begin{equation*}
    \forall\,i:\quad
    \mathbb{E}_{\mathcal{X}}\bigl[U_i(\widetilde{\mathbf{x}_i})\bigr]\;\ge\;0
  \end{equation*}

  \vspace{1ex}
  \item \textbf{Interim Individually Rational}:\\
  各プレイヤー \(i\)が,自分のtypeの実現値を知っているが,他人のtypeの実現値を知らない時に,すべての参加者において,真実申告をしたときの、期待効用が非負である場合.
  \begin{equation*}
    \forall\,i,\;\forall\,x_i\in\mathcal{X}_i:\quad
    \mathbb{E}_{\mathcal{X}_{-i}}\bigl[Q_i(x_i,\mathbf{x}_{-i})\,x_i - M_i(x_i,\mathbf{x}_{-i})\bigr]\;\ge\;0
  \end{equation*}

  \vspace{1ex}
  \item \textbf{Ex\textendash{}post Individually Rational}:\\
  各プレイヤー \(i\)が,全ての人のtypeが公開されていて知っている時に,すべての参加者において,真実申告をしたときの効用が非負である場合.
  \begin{equation*}
    \forall\,i,\;\forall\,x_i:\quad
    Q_i(x_i,\mathbf{x}_{-i})\,x_i - M_i(x_i,\mathbf{x}_{-i})\;\ge\;0
  \end{equation*}
  \end{enumerate}
\end{frame}

\begin{frame}{2.1  Preliminaries:最適オークションの設計}
  \small
  \begin{itemize}
    \item DSICであるオークションにおいては,各参加者が真実を申告するのが最善となり,均衡である.
    \item したがって,ある評価の組 $\mathbf{v}$ に対する均衡での収入は,単に次の和として表される:
    \[
      \sum_i p_i(\mathbf{v})
    \]
    \item 最適オークション設計(Optimal Auction Design)の目的は,期待収入を最大化する DSIC オークションを見つけること:
    \[
      \sum_{i} \mathbb{E}\Bigl[p_i(\mathbf{v})\Bigr]
    \]
    \item Düttingでは,BIC ではなく DSIC オークションに焦点を当てる.
    \item 評価の分布や合理性に関する Common Knowledge がなくても,DSIC オークションでは真実の申告が均衡となるから.
  \end{itemize}
\end{frame}


\end{document}

