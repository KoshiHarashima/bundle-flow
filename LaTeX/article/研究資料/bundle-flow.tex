\documentclass[a4paper]{article}

% ==== article 用(upLaTeX + dvipdfmx 前提)====
% Overleaf/latexmk で uplatex→dvipdfmx を想定
% エンジンはクラスやパッケージで自動判別しない(手元設定で固定)

% --- 数式・基本 ---
\usepackage{amsmath,amssymb,amsthm,mathtools,bm}
\allowdisplaybreaks
\numberwithin{equation}{section}

% --- 図とハイパーリンク(dvipdfmx明示)---
\usepackage[dvipdfmx]{graphicx}
\usepackage[dvipdfmx,hidelinks,bookmarks=true,bookmarksnumbered=true]{hyperref}
\usepackage[nameinlink,capitalise]{cleveref}

% --- 表(経済論文向け)---
\usepackage{booktabs,threeparttable,threeparttablex}
\usepackage{tabularx,array}
\renewcommand{\arraystretch}{1.1}

% --- 色・箇条書き ---
\usepackage{xcolor}
\usepackage[shortlabels]{enumitem}
\setlist{noitemsep,topsep=2pt}

% --- キャプション ---
\usepackage{caption}
\usepackage{subcaption}

% --- 日本語(upLaTeX)---
\usepackage{otf}
\usepackage[noalphabet]{pxchfon}
\renewcommand{\kanjifamilydefault}{\gtdefault}
% HaranoAji が無い環境では次行をコメントアウト
\IfFileExists{HaranoAjiGothic-Medium.otf}{%
  \setboldgothicfont{HaranoAjiGothic-Medium.otf}%
}{}

% 図の相対パス(article配下の figures を既定に)
\graphicspath{{./figures/}}

% ==== 定理環境(節ごと番号)====
\theoremstyle{plain}
\newtheorem{theorem}{Theorem}[section]
\newtheorem{lemma}[theorem]{Lemma}
\newtheorem{proposition}[theorem]{Proposition}
\newtheorem{corollary}[theorem]{Corollary}
\theoremstyle{definition}
\newtheorem{definition}[theorem]{Definition}
\newtheorem{assumption}[theorem]{Assumption}
\newtheorem{problem}[theorem]{Problem}
\theoremstyle{remark}
\newtheorem{remark}[theorem]{Remark}

% cleveref の表示名
\crefname{theorem}{Theorem}{Theorems}
\crefname{lemma}{Lemma}{Lemmas}
\crefname{proposition}{Proposition}{Propositions}
\crefname{corollary}{Corollary}{Corollaries}
\crefname{definition}{Definition}{Definitions}
\crefname{assumption}{Assumption}{Assumptions}
\crefname{problem}{Problem}{Problems}
\crefname{remark}{Remark}{Remarks}
\crefname{equation}{Eq.}{Eqs.}
\crefname{figure}{Fig.}{Figs.}
\crefname{table}{Table}{Tables}
\crefname{section}{Section}{Sections}

%======数学記法===============
\DeclareMathOperator*{\argmax}{arg\,max}


% ---- 文献(biblatex + biber 前提;Paperpileの .bib を主ファイル基準で)----
\usepackage{csquotes}
\usepackage[
  backend=biber,
  style=apa,      % 必要に応じて phys/authoryear へ差替え
  natbib=true
]{biblatex}
\addbibresource{paperpile.bib}


% ==== メタ情報(必要に応じて毎回ここだけ更新)====
\newcommand{\papertitle}{N}
\newcommand{\paperauthor}{Koshi Harashima}
\newcommand{\paperaffil}{Northwestern University}
\newcommand{\paperdate}{October 6, 2025}

% ==== 数式系のマクロ ====
\newcommand*{\N}{\mathbb{N}}
\newcommand*{\Z}{\mathbb{Z}}
\newcommand*{\Q}{\mathbb{Q}}
\newcommand*{\R}{\mathbb{R}}
\newcommand*{\C}{\mathbb{C}}
\newcommand*{\I}{\mathbb{I}}

\newcommand*{\boldalpha}  {\boldsymbol \alpha}
\newcommand*{\boldbeta}   {\boldsymbol \beta}
\newcommand*{\boldgamma}  {\boldsymbol \gamma}
\newcommand*{\bolddelta}  {\boldsymbol \delta}
\newcommand*{\boldepsilon}{\boldsymbol \epsilon}
\newcommand*{\boldtheta}  {\boldsymbol \theta}

\DeclareMathOperator*{\esssup}{ess\,sup}


\renewcommand*{\bibfont}{\small}
\setlength\bibitemsep{0.1\baselineskip}

\usepackage{enumitem}
\setlist{nosep}

\title{Bundle-Flow}
\author{\paperauthor \\ \paperaffil}
\date{October 9, 2025}

\begin{document}

\maketitle

\section*{Formulation of Combinational Auction}
\textbf{Mechanism}
\begin{itemize}
    \item まず商品集合Mとバンドルの集合Sを以下のように定義する(この時、1商品もバンドルとして扱う).:
    \begin{equation*}
        M=\{1,\dots,m\},\qquad \mathcal{S}:=2^{M}.
    \end{equation*}
    \item $v$ は各バンドルの価値関数で,そのタイプ空間 $V$ 上に分布 $F$ が与えられます.
    \begin{equation*}
        v:\ \mathcal{S}\to\mathbb{R}_{\ge 0},\qquad v\sim F \ \text{on}\ V.
    \end{equation*}
    \item Revelation Principleより、分析を直接メカニズム$M(g, p)$に制限しても一般性を失わない(Mas-Colell et al.(1995))。$X$ はバンドル上のくじ(確率配分)の集合で,$g$ が配分ルール,$p$ が支払いルールである。
    \begin{equation*}
        X:=\Delta(\mathcal{S}),\qquad g:V\to X,\qquad p:V\to\mathbb{R}_{\ge 0}.
    \end{equation*}
    \item 効用は準線形かつリスク中立で,申告 $b$ に対し $u(v;b)$ を得る.
    \begin{equation*}
    u(v;b)=v\bigl(g(b)\bigr)-p(b).
    \end{equation*}
    \item DSICとは以下の式が成り立つことである。
    \begin{equation*}
    \forall v,b\in V:\quad u(v;v)\ \ge\ u(v;b).
    \end{equation*}  
\end{itemize}
% MyersonとRochetの話とは接続できないのか。
\textbf{Menu representation(3.1)}
\begin{itemize}
    \item メニュー $B$ は $K$ 個の要素からなり,各要素はbundleの確率分布と価格の組である。
    \begin{equation*}
        B=\bigl(B^{(1)},\dots,B^{(K)}\bigr),\qquad B^{(k)}=\bigl(\alpha^{(k)},\beta^{(k)}\bigr).
    \end{equation*}
    \item $\alpha^{(k)}$ はバンドル上の確率分布,$\beta^{(k)}$ は要素 $k$ の価格である
    \begin{equation*}
        \alpha^{(k)}:\ \mathcal{S}\to[0,1],\qquad \sum_{S\subseteq M}\alpha^{(k)}(S)=1,\qquad \beta^{(k)}\in\mathbb{R}_{\ge 0}.
    \end{equation*}
    \item 申告 $b$ に対し,期待価値から価格を引いた効用が最大となるメニュー要素を選ぶ.
    \begin{equation*}
        k^\ast(b)\ \in\ \argmax_{k=1,\dots,K}\ \sum_{S\subseteq M}\alpha^{(k)}(S)\,b(S)\ -\ \beta^{(k)}.
    \end{equation*}
\end{itemize}
\textbf{Flow-based Bundle Generation}
\begin{itemize}
    \item 各メニュー要素 $k$ ごとに,連続表現空間でバンドルの潜在ベクトル $\bm{s}^{(k)}_t$ をベクトル場 $\bm{\phi}^{(k)}$ に沿って流します.
    \begin{equation*}
        d\,\bm{s}^{(k)}_t\ =\ \bm{\phi}^{(k)}\!\bigl(t,\bm{s}^{(k)}_t\bigr),\qquad t\in[0,1],\ \ \bm{s}^{(k)}_t\in\mathbb{R}^m.
    \end{equation*}
    \item 初期分布 $\alpha_0^{(k)}$ をフロー写像 $\Phi^{(k)}_{T}$ で押し出し(pushforward),threshold rounding map $\mathcal{R}:\mathbb{R}^m\to\mathcal{S}$(買う/買わない)でdiscrete bundlesへ写して得た終端のbundle分布が $\alpha_T^{(k)}$ です
    \begin{equation*}
        \alpha_T^{(k)}\ =\ \mathcal{R}_{\sharp}\Bigl(\,\Phi^{(k)}_{T}\,\Bigr)_{\sharp}\,\alpha_0^{(k)}.
    \end{equation*}
    \item $\rho^{(k)}_t$ は連続空間での密度で,その時間発展はLiouville方程式に従い,学習に用いることができます
    \begin{equation*}
    \partial_t\rho^{(k)}_t\ +\ \nabla\!\cdot\!\bigl(\rho^{(k)}_t\,\bm{\phi}^{(k)}\bigr)\ =\ 0.
    \end{equation*}
\end{itemize}
\textbf{Two-step learning}
\[
\begin{aligned}
\text{Stage 1:}\quad
&\bm{\phi}^{(k)}\text{ を学習し、}
\{0,1\}^m\text{ 近傍の }\mathcal{S}\ (\text{feasible bundles})
\text{ が広く到達可能となるようにする.}\\
\text{Stage 2:}\quad
&\bm{\phi}^{(k)}\text{ を固定し、}
\bigl(\alpha_0^{(k)},\beta^{(k)}\bigr)\text{ を最適化して期待収益を最大化する.}
\end{aligned}
\]


\nocite{*}
\printbibliography

\end{document}
