\documentclass[dvipdfmx,autodetect-engine]{article}

% ==== article 用(upLaTeX + dvipdfmx 前提)====
% Overleaf/latexmk で uplatex→dvipdfmx を想定
% エンジンはクラスやパッケージで自動判別しない(手元設定で固定)

% --- 数式・基本 ---
\usepackage{amsmath,amssymb,amsthm,mathtools,bm}
\allowdisplaybreaks
\numberwithin{equation}{section}

% --- 図とハイパーリンク(dvipdfmx明示)---
\usepackage[dvipdfmx]{graphicx}
\usepackage[dvipdfmx,hidelinks,bookmarks=true,bookmarksnumbered=true]{hyperref}
\usepackage[nameinlink,capitalise]{cleveref}

% --- 表(経済論文向け)---
\usepackage{booktabs,threeparttable,threeparttablex}
\usepackage{tabularx,array}
\renewcommand{\arraystretch}{1.1}

% --- 色・箇条書き ---
\usepackage{xcolor}
\usepackage[shortlabels]{enumitem}
\setlist{noitemsep,topsep=2pt}

% --- キャプション ---
\usepackage{caption}
\usepackage{subcaption}

% --- 日本語(upLaTeX)---
\usepackage{otf}
\usepackage[noalphabet]{pxchfon}
\renewcommand{\kanjifamilydefault}{\gtdefault}
% HaranoAji が無い環境では次行をコメントアウト
\IfFileExists{HaranoAjiGothic-Medium.otf}{%
  \setboldgothicfont{HaranoAjiGothic-Medium.otf}%
}{}

% 図の相対パス(article配下の figures を既定に)
\graphicspath{{./figures/}}

% ==== 定理環境(節ごと番号)====
\theoremstyle{plain}
\newtheorem{theorem}{Theorem}[section]
\newtheorem{lemma}[theorem]{Lemma}
\newtheorem{proposition}[theorem]{Proposition}
\newtheorem{corollary}[theorem]{Corollary}
\theoremstyle{definition}
\newtheorem{definition}[theorem]{Definition}
\newtheorem{assumption}[theorem]{Assumption}
\newtheorem{problem}[theorem]{Problem}
\theoremstyle{remark}
\newtheorem{remark}[theorem]{Remark}

% cleveref の表示名
\crefname{theorem}{Theorem}{Theorems}
\crefname{lemma}{Lemma}{Lemmas}
\crefname{proposition}{Proposition}{Propositions}
\crefname{corollary}{Corollary}{Corollaries}
\crefname{definition}{Definition}{Definitions}
\crefname{assumption}{Assumption}{Assumptions}
\crefname{problem}{Problem}{Problems}
\crefname{remark}{Remark}{Remarks}
\crefname{equation}{Eq.}{Eqs.}
\crefname{figure}{Fig.}{Figs.}
\crefname{table}{Table}{Tables}
\crefname{section}{Section}{Sections}

%======数学記法===============
\DeclareMathOperator*{\argmax}{arg\,max}


% ---- 文献(biblatex + biber 前提;Paperpileの .bib を主ファイル基準で)----
\usepackage{csquotes}
\usepackage[
  backend=biber,
  style=apa,      % 必要に応じて phys/authoryear へ差替え
  natbib=true
]{biblatex}
\addbibresource{paperpile.bib}


% ==== メタ情報(必要に応じて毎回ここだけ更新)====
\newcommand{\papertitle}{N}
\newcommand{\paperauthor}{Koshi Harashima}
\newcommand{\paperaffil}{Northwestern University}
\newcommand{\paperdate}{October 6, 2025}

% ==== 数式系のマクロ ====
\newcommand*{\N}{\mathbb{N}}
\newcommand*{\Z}{\mathbb{Z}}
\newcommand*{\Q}{\mathbb{Q}}
\newcommand*{\R}{\mathbb{R}}
\newcommand*{\C}{\mathbb{C}}
\newcommand*{\I}{\mathbb{I}}

\newcommand*{\boldalpha}  {\boldsymbol \alpha}
\newcommand*{\boldbeta}   {\boldsymbol \beta}
\newcommand*{\boldgamma}  {\boldsymbol \gamma}
\newcommand*{\bolddelta}  {\boldsymbol \delta}
\newcommand*{\boldepsilon}{\boldsymbol \epsilon}
\newcommand*{\boldtheta}  {\boldsymbol \theta}

\DeclareMathOperator*{\esssup}{ess\,sup}


\title{\papertitle}
\author{\paperauthor \\ \paperaffil}
\date{\paperdate}

\begin{document}
\maketitle

\section{はじめに}

\section*{Radon 測度}
\paragraph{定義}
\begin{itemize}
  \item \(X\) を局所コンパクト Hausdorff,\(\mathcal{B}(X)\) を Borel \(\sigma\)-代数とする。
  \item \(\mu\) は各コンパクト \(K\subset X\) で \(\mu(K)<\infty\)(局所有限)。
  \item 内部正則:\(\mu(A)=\sup\{\mu(K):K\subset A,\ K\ \text{compact}\}\)。
  \item 外部正則:\(\mu(A)=\inf\{\mu(U):A\subset U,\ U\ \text{open}\}\)。
\end{itemize}
\paragraph{直感}
\begin{itemize}
  \item 質量を開集合とコンパクト集合で「挟んで」測れる Borel 測度。
  \item \(\mathbb{R}^n\) では Lebesgue・表面測度・Dirac を自然に含む。
  \item 積分・確率分布の基準測度として扱いやすい。
\end{itemize}
\paragraph{例}
\begin{itemize}
  \item Dirac:\(\delta_{x_0}(A)=\mathbf{1}_A(x_0)\)。
  \item 区間の一様:\(U_{[a,b]}(E)=\int_E \frac{1}{b-a}\mathbf{1}_{[a,b]}(x)\,d\lambda(x)\)。
  \item 表面測度:滑らかな \(S\subset\mathbb{R}^n\) に対し \(\sigma_S(A)=\mathcal{H}^{n-1}(A\cap S)\)。
\end{itemize}
\paragraph{Regularity}
\begin{itemize}
  \item \textbf{Bottom line:} A Radon measure is a Borel measure finite on compacts and \emph{regular}: you can approximate sets from inside by compacts and from outside by opens. Equivalently,
  \[
  \forall A\in\mathcal{B}(X),\ \forall \varepsilon>0,\ \exists K\subset A\subset U\ (K\ \text{compact},\ U\ \text{open})\ \text{s.t.}\ \mu(U\setminus K)<\varepsilon.
  \]
  \item \textbf{Intuition:} Most mass sits in a compact core; any set can be covered as tightly as you like by open cushions. This stabilizes integration on \(C_c(X)\) (Riesz) and behaves well under continuous pushforwards.
\end{itemize}

\paragraph{Examples}
\begin{itemize}
  \item Dirac \(\delta_{x_0}\): choose \(K=\{x_0\}\), \(U\) any neighborhood of \(x_0\).
  \item Uniform on \([a,b]\): use compact \(K\subset[a,b]\) and slightly larger open \(U\supset[a,b]\).
  \item Surface measure \(\sigma_S\): take compact patches \(K\subset S\) and tubular neighborhoods \(U\supset S\).
\end{itemize}

\paragraph{Quick facts}
\begin{itemize}
  \item On metric/Polish spaces (e.g. \(\mathbb{R}^n\)), finite Borel measures are regular \(\Rightarrow\) probabilities are Radon.
  \item Regularity is preserved by continuous pushforwards \(T_\#\mu\).
\end{itemize}


\vspace{1em}

\section*{符号付測度}
\paragraph{定義}
\begin{itemize}
  \item 有限符号付測度:\(\nu:\mathcal{A}\to\mathbb{R}\) が可算加法的,\(\nu(\emptyset)=0\)。
  \item 一般には \([-\infty,\infty]\) 値も許すが \(+\infty,-\infty\) の同時出現は不可。
  \item 全変動:\(|\nu|(A)=\sup\sum_i |\nu(A_i)|\)(可算分割に沿う上限)。
\end{itemize}
\paragraph{直感}
\begin{itemize}
  \item プラス・マイナスの質量を許す;しばしば \(\nu=\mu_1-\mu_2\) と見なせる。
  \item \(\int f\,d\nu\) は「符号付き重み」での平均。
\end{itemize}
\paragraph{例}
\begin{itemize}
  \item \(\nu=\delta_{1}-\delta_{0}\)。
  \item \(\nu(E)=\int_E w(x)\,d\lambda(x)\)(\(w\in L^1\) で符号可)。
  \item 幾何的差:\(\nu=\lambda-\sigma_S\) など。
\end{itemize}

\vspace{1em}

\section*{Jordan 分解}
\paragraph{定義}
\begin{itemize}
  \item 有限符号付測度 \(\nu\) に一意な \(\nu^+,\nu^-\ge0\) が存在し \(\nu=\nu^+-\nu^-\)。
  \item \(\nu^+\perp \nu^-\),全変動は \(|\nu|=\nu^++\nu^-\);Hahn 分解に対応。
\end{itemize}
\paragraph{直感}
\begin{itemize}
  \item 正と負の山を分離する標準形で,議論を正測度に還元できる。
  \item 以後は \(\nu^+,\nu^-\) を個別に扱えば良い。
\end{itemize}
\paragraph{例}
\begin{itemize}
  \item \(\delta_{1}-\delta_{0}\Rightarrow(\nu^+,\nu^-)=(\delta_{1},\delta_{0})\)。
  \item \(\nu=f\mu\Rightarrow \nu^+=f^+\mu,\ \nu^-=f^-\mu\)(\(f^\pm=\max(\pm f,0)\))。
\end{itemize}

\vspace{1em}

\section*{Radon--Nikodym 定理}
\paragraph{定義}
\begin{itemize}
  \item \(\sigma\)-finite な \(\mu,\nu\) で \(\nu\ll\mu\) なら \(\exists f\in L^1(\mu)\) s.t. \(\nu(E)=\int_E f\,d\mu\)。
  \item \(f\) は \(\mu\)-a.e. 一意で \(\dfrac{d\nu}{d\mu}\)(Radon--Nikodym 微分)と書く。
\end{itemize}
\paragraph{直感}
\begin{itemize}
  \item 絶対連続な測度は「密度」\(f\) をもち,\(d\nu=f\,d\mu\) と表せる。
  \item 確率では密度・尤度比;解析では変数変換の核。
\end{itemize}
\paragraph{例}
\begin{itemize}
  \item \(U_{[a,b]}\ll\lambda\),\(\dfrac{dU_{[a,b]}}{d\lambda}=\dfrac{1}{b-a}\mathbf{1}_{[a,b]}\)。
  \item \(\nu(E)=\int_E w\,d\lambda\Rightarrow \dfrac{d\nu}{d\lambda}=w\)。
  \item \(\delta_{x_0}\not\ll\lambda\)(密度は存在しない)。
\end{itemize}

\vspace{1em}

\section*{5問小テスト}
\begin{enumerate}
  \item Radon 測度の内部正則・外部正則を一文で述べよ。
  \item \(\delta_{0}\) は \(\lambda\) に絶対連続か。最短の理由を挙げよ。
  \item \(\nu=\delta_{1}-\delta_{0}\) の Jordan 分解 \((\nu^+,\nu^-)\) を求めよ。
  \item \(\nu(E)=\int_E \big(2\mathbf{1}_{[0,1]}-\mathbf{1}_{[1,2]}\big)\,d\lambda\) の \(\dfrac{d\nu}{d\lambda}\) を書け。
  \item \(T(x)=ax+b\,(a\neq0)\)。\(\mu=f\lambda\) として \(T_\#\mu\) の \(\lambda\) に関する密度を与えよ。
\end{enumerate}

\vspace{1em}

\section*{transformed measure(変換測度)への接続}
可測写像 \(T:X\to Y\) に対し \(T_\#\mu(A):=\mu(T^{-1}(A))\) は(\(T\) が連続なら)Radon を保ち,符号付でも \(T_\#\nu=T_\#\nu^+-T_\#\nu^-\) と Jordan 分解が可換。さらに \(\nu\ll\mu\) で \(\nu=\tfrac{d\nu}{d\mu}\,\mu\) なら \(T_\#\nu\ll T_\#\mu\)。特に \(T\) が \(C^1\) 微分同相で \(\mu=f\,\lambda\)(Lebesgue)なら変数変換より
\[
\frac{d\,(T_\#(f\lambda))}{d\lambda}(y)=f\big(T^{-1}y\big)\,|\det DT^{-1}(y)|,
\]
すなわち「transformed measure」の密度は元の密度の合成にヤコビアンが掛かる。

\end{document}