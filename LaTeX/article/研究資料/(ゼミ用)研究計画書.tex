\documentclass[dvipdfmx,autodetect-engine]{article}

% ==== article 用(upLaTeX + dvipdfmx 前提)====
% Overleaf/latexmk で uplatex→dvipdfmx を想定
% エンジンはクラスやパッケージで自動判別しない(手元設定で固定)

% --- 数式・基本 ---
\usepackage{amsmath,amssymb,amsthm,mathtools,bm}
\allowdisplaybreaks
\numberwithin{equation}{section}

% --- 図とハイパーリンク(dvipdfmx明示)---
\usepackage[dvipdfmx]{graphicx}
\usepackage[dvipdfmx,hidelinks,bookmarks=true,bookmarksnumbered=true]{hyperref}
\usepackage[nameinlink,capitalise]{cleveref}

% --- 表(経済論文向け)---
\usepackage{booktabs,threeparttable,threeparttablex}
\usepackage{tabularx,array}
\renewcommand{\arraystretch}{1.1}

% --- 色・箇条書き ---
\usepackage{xcolor}
\usepackage[shortlabels]{enumitem}
\setlist{noitemsep,topsep=2pt}

% --- キャプション ---
\usepackage{caption}
\usepackage{subcaption}

% --- 日本語(upLaTeX)---
\usepackage{otf}
\usepackage[noalphabet]{pxchfon}
\renewcommand{\kanjifamilydefault}{\gtdefault}
% HaranoAji が無い環境では次行をコメントアウト
\IfFileExists{HaranoAjiGothic-Medium.otf}{%
  \setboldgothicfont{HaranoAjiGothic-Medium.otf}%
}{}

% 図の相対パス(article配下の figures を既定に)
\graphicspath{{./figures/}}

% ==== 定理環境(節ごと番号)====
\theoremstyle{plain}
\newtheorem{theorem}{Theorem}[section]
\newtheorem{lemma}[theorem]{Lemma}
\newtheorem{proposition}[theorem]{Proposition}
\newtheorem{corollary}[theorem]{Corollary}
\theoremstyle{definition}
\newtheorem{definition}[theorem]{Definition}
\newtheorem{assumption}[theorem]{Assumption}
\newtheorem{problem}[theorem]{Problem}
\theoremstyle{remark}
\newtheorem{remark}[theorem]{Remark}

% cleveref の表示名
\crefname{theorem}{Theorem}{Theorems}
\crefname{lemma}{Lemma}{Lemmas}
\crefname{proposition}{Proposition}{Propositions}
\crefname{corollary}{Corollary}{Corollaries}
\crefname{definition}{Definition}{Definitions}
\crefname{assumption}{Assumption}{Assumptions}
\crefname{problem}{Problem}{Problems}
\crefname{remark}{Remark}{Remarks}
\crefname{equation}{Eq.}{Eqs.}
\crefname{figure}{Fig.}{Figs.}
\crefname{table}{Table}{Tables}
\crefname{section}{Section}{Sections}

%======数学記法===============
\DeclareMathOperator*{\argmax}{arg\,max}


% ---- 文献(biblatex + biber 前提;Paperpileの .bib を主ファイル基準で)----
\usepackage{csquotes}
\usepackage[
  backend=biber,
  style=apa,      % 必要に応じて phys/authoryear へ差替え
  natbib=true
]{biblatex}
\addbibresource{paperpile.bib}


% ==== メタ情報(必要に応じて毎回ここだけ更新)====
\newcommand{\papertitle}{N}
\newcommand{\paperauthor}{Koshi Harashima}
\newcommand{\paperaffil}{Northwestern University}
\newcommand{\paperdate}{October 6, 2025}

% ==== 数式系のマクロ ====
\newcommand*{\N}{\mathbb{N}}
\newcommand*{\Z}{\mathbb{Z}}
\newcommand*{\Q}{\mathbb{Q}}
\newcommand*{\R}{\mathbb{R}}
\newcommand*{\C}{\mathbb{C}}
\newcommand*{\I}{\mathbb{I}}

\newcommand*{\boldalpha}  {\boldsymbol \alpha}
\newcommand*{\boldbeta}   {\boldsymbol \beta}
\newcommand*{\boldgamma}  {\boldsymbol \gamma}
\newcommand*{\bolddelta}  {\boldsymbol \delta}
\newcommand*{\boldepsilon}{\boldsymbol \epsilon}
\newcommand*{\boldtheta}  {\boldsymbol \theta}

\DeclareMathOperator*{\esssup}{ess\,sup}


\newcommand{\Pow}{\mathcal{P}}
\newcommand{\B}{\mathcal{B}}
\newcommand{\conv}{\mathcal{conv}}
\newcommand{\inp}{\mathcal{inp}}
\newcommand{\Chi}{\mathcal{Chi}}


\title{研究計画書}
\author{\paperauthor \\ \paperaffil}
\date{\paperdate}

\begin{document}

\maketitle

\section*{Research Question}
多商品のオークションの理論的な解明。特に商品バンドルとメニューの組み合わせについての解析。

\section*{Motivation}


\section*{Literature Review}

\section*{Model}

\paragraph{Primitives.}
I owe my notation to Feng(2025).
\begin{itemize}
    \item Fix $n\ge 2$ with index set $N:=\{1,\dots,n\}$.
    \item Let the set of all bundles be $\B$
    \item The seller commits to a feasible bundle family $B \subseteq \B$ that includes the outside option $\emptyset$ and singletons.
    \footnote{Grand bundling $\B=\{\emptyset,N\}$. Nested bundling: $\B$ is \emph{laminar} , which satisfy 
        \[
            \forall\, b,b' \in \mathcal{B}, \quad 
            \bigl(b \subseteq b'\bigr) \ \lor\ 
            \bigl(b' \subseteq b\bigr) \ \lor\ 
            \bigl(b \cap b' = \emptyset\bigr).
        \]}
    \item The buyer has a type $t \in \mathcal{T} \subseteq \mathbb{R}^d$ and a valuation function
    \[
        v_t : \mathcal{B} \to \mathbb{R}_{\ge 0}, \qquad b \mapsto v_t(b),
    \]
    with $v_t(\emptyset) = 0$.
    \item Each coordinate $x_j \in X_j = [L_j, H_j]$.
    \item The type space is the hyperrectangle $\mathcal{X}= \prod_{j=1}^n X_j$.
    \item The seller knows a probability distribution over $\mathcal{X}$ with an almost everywhere three-times differentiable density $f$.
\end{itemize}

\paragraph{Mechanisms.}
A (direct-revelation) mechanism is a pair $(a,p)$ where
\[
    a:\mathcal{X}\to \Delta(\B),\qquad p:\mathcal{X}\to \R_{\ge 0}.
\]
For a report $\hat t$, $a(\hat t)$ is a probability distribution over bundles:

\[
    a(\hat t)=\big(a_b(\hat t)\big)_{b\in\B}, \quad
    \sum_{b\in\B}a_b(\hat t)=1, \quad
    a_b(\hat t)\ge 0.
\]
The (quasi-linear) utility of true type $t$ when reporting $\hat t$ is
\[
    u(\hat t;t)\;=\;\sum_{b\in\B} a_b(\hat t)\,v_t(b)\;-\;p(\hat t).
\]
Thus, Expected revenue is $\mathbb{E}_{x\sim F}\!\big[p(x)\big]$.

\paragraph{Constraints.}There are two constraints: IR and IC.\\
\emph{Individual Rationality (IR):} 
\[
u(x;x)\ge 0 \quad
\]
\emph{Incentive Compatibility (IC):} 
\[
u(t;t)\ge u(\hat t;t) \quad \forall \hat x, x\in X
\]

\paragraph{Program.}Thus, \\
The seller's problem can be constructed as following:
\[
\max_{(a,p)}\ \mathbb{E}_{x\sim F}\!\big[p(x)\big]\quad\text{s.t.\ IC, IR, and }a(\cdot)\in\Delta(B).
\]

\subsection*{From bundle probabilities to 
item marginals}
For any lottery $a(\hat t)$, define item-wise marginals
\[
q_i(\hat t)\coloneqq \sum_{b\in\B:\ i\in b} a_b(\hat t)\in[0,1],\qquad i=1,\dots,n.
\]
Let $\chi(b)\in\{0,1\}^n$ be the incidence vector of bundle $b$.
The \emph{bundle polytope} is
\[
P(\B)\coloneqq \conv\big\{\,\chi(b)\ :\ b\in\B\,\big\}\ \subseteq\ [0,1]^n.
\]

\begin{proposition}[Feasible marginals]
For any report $\hat t$, the induced marginals $q(\hat t)=(q_i(\hat t))_{i=1}^n$ belong to $P(\B)$.
Conversely, a vector $q\in[0,1]^n$ is implementable by some lottery over $\B$ iff $q\in P(\B)$.
\end{proposition}

\begin{remark}[When can we replace bundle lotteries by item marginals?]
If valuations are \emph{additive}, $v_t(b)=\sum_{i\in b} t_i$, then
\[
\sum_{b\in\B} a_b(\hat t)\,v_t(b)=\sum_{i=1}^n q_i(\hat t)\,t_i,
\]
so the buyer's expected value depends only on $q(\hat t)$; a per-item description is lossless \emph{subject to} the feasibility set $P(\B)$.
For unit-demand or general (non-additive) $v_t$, the expectation depends on the \emph{joint} distribution over bundles, so marginals $q$ do not suffice in general.
\end{remark}

\subsection*{Unit-demand (UD) specialization}
Suppose $d=n$ and type $x=(x_1,\dots,x_n)\in X:=\prod_{j=1}^n [L_j,H_j]\subset \R_{\ge 0}^n$. Define
\[
v_x(S)=\max_{j\in S} x_j,\qquad v_x(\emptyset)=0.
\]
W.l.o.g., restrict $\B=\{\emptyset,\{1\},\dots,\{n\}\}$; i.e., allocate at most one item (plus outside option).
Write $a(x)=(a_0(x),a_1(x),\dots,a_n(x))\in\Delta(\{0,1,\dots,n\})$ with $a_0$ for ``no allocation''.
The buyer's utility at report $\hat x$ under true $x$ is
\[
u(\hat x;x)=\sum_{j=1}^n a_j(\hat x)\,x_j - p(\hat x).
\]

\begin{theorem}[Rochet (1985) characterization in the UD case]
A mechanism $(a,p)$ is IC iff there exists a convex function $U:X\to\R$ such that for almost every $x$,
\[
a_{1:n}(x)\in \partial U(x)\quad\text{and}\quad p(x)=\inp{a_{1:n}(x)}{x}-U(x),
\]
with $a_0(x)=1-\sum_{j=1}^n a_j(x)\in[0,1]$.
IR is equivalent to $U(x)\ge 0$ for all $x\in X$.
\end{theorem}

\subsection*{Grand and nested bundling within the template}

\begin{itemize}
  \item \textbf{Grand bundling:} take $\B=\{\emptyset,N\}$ and any $v_t$.
  Then $a(t)$ is a Bernoulli over $\{0,N\}$, and 
  \[
    u(\hat t;t) = a_N(\hat t)v_t(N) - p(\hat t).
  \]
  \item \textbf{Nested bundling:} take $\B$ laminar (tree-structured).
  The mechanism still uses $a:\mathcal{T}\to\Delta(\B)$; IC/IR remain as above.
  Feasible marginals $q(\cdot)$ must lie in $P(\B)=\conv\{\chi(b):b\in\B\}$.
  With laminar $\B$, $P(\B)$ has additional structure exploitable algorithmically.
\end{itemize}


\paragraph{Per-item vs.\ per-bundle probabilities.}
Per-item $q(\cdot)$ is only a \emph{summary} of $a(\cdot)$:
\[
\text{Bundle level: } a(\cdot)\in \Delta(\B)\quad\Longrightarrow\quad
\text{marginals } q(\cdot)\in P(\B).
\]
Additive $v_t$ $\Rightarrow$ $q$ is sufficient for payoffs; UD/non-additive $v_t$ $\Rightarrow$ the joint law $a$ matters.
\section*{Analysis}


\section*{Simulation/Validation}


\section*{Discussion/Implication}

\section*{Research Plan}

\section*{Reference}


\nocite{*}
\printbibliography
\end{document}